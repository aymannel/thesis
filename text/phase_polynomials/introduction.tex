\chapter{Phase Polynomials}%
\label{phase-polynomials}

Recalling that phase gadgets are diagonal in the computational $Z$ basis, we call a set of Pauli gadgets diagonal when it consists only of phase gadgets \cite{Cowtan2020}. Such sets of phase gadgets are known as \textit{phase polynomials} and are themselves diagonal in the computational $Z$ basis \cite{Cowtan2019}. Phase polynomial synthesis then refers to finding a quantum circuit that optimally implements some phase polynomial. There are several well-known phase polynomial synthesis algorithms \cite{Amy2013}, \cite{Amy2014}, \cite{Nam2018}, \cite{Maslov2018}.

As in Cowtan \textit{et al} \cite{Cowtan2020}, a set of Pauli gadgets $S$ can be simultaneously diagonalised by a Clifford subcircuit $C$ when all of the Pauli gadgets in the set commute. That is, by conjugating a set of commuting Pauli gadgets, we can re-express the circuit as some phase polynomial that can later be synthesised in some optimal way.

\includezxdiagram{phase_polynomials/phase_polynomial}{0.9}

As we will see in Chapter \ref{excitation-operators}, the excitation operators used in the DISCO-VQE algorithm consist of commuting sets of Pauli gadgets. Therefore, there exists some Clifford subcircuit $C$ that diagonalise these excitation operators. As stated in Cowtan \textit{et al} \cite{Cowtan2020}, whilst diagonalisation may incur gate overhead, in practice, the reduction in circuit depth arising from synthesising the resulting phase polynomial more than makes up for the overhead.
