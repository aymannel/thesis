A central challenge in computational quantum chemistry is the accurate simulation of fermionic systems. At the heart of these calculations lies the need to solve the Schrödinger equation to determine the many-electron wavefunction. An exact solution to this problem scales exponentially with the number of electrons. Classical computers struggle to store the increasingly large wavefunctions making this problem computationally intractable in many cases. In contrast, gate-based quantum computing presents a promising solution, offering the potential to represent electronic wavefunctions with polynomially scaling resources \cite{Burton2023}. In other words, quantum computers are a natural tool of choice for simulating processes that are inherently quantum \cite{Yeung2020}.

In the last two decades many advancements in quantum computing have been made in both hardware and software bringing us closer to being able to simulate molecular systems. Despite these advancements, we remain in the so-called Noisy Intermediate Scale Quantum (NISQ) era, characterised by challenges such as poor qubit fidelity, low qubit connectivity and limited coherence times. The NISQ era represents a transitional phase in quantum computing, where quantum devices are not yet error-corrected but are still capable of performing computations beyond the reach of classical computers. Overcoming the limitations of the NISQ era is crucial for realising the full potential of quantum computing in various fields, including quantum chemistry and materials science.

The Variational Quantum Eigensolver (VQE) algorithm is a method used to estimate the ground state energy of a molecular Hamiltonian by preparing a trial wavefunction, calculating its energy, and optimising the wavefunction parameters classically until the energy converges to the optimal approximation for the ground state energy \cite{McClean2016}. It is recognised as a leading algorithm for quantum simulation on NISQ devices due to its reduced resource requirements in terms of qubit count and coherence time \cite{Kirby2020}.

This thesis extends methods developed by Richie Yeung \cite{Yeung2020} for the preparation and analysis of parametrised quantum circuits, and applies them to ansätze representing fermionic wavefunctions. We are concerned with two main questions along this theme. Firstly, can we use the ZX calculus [cite] to gain insights about the structure of the unitary product ansatz in the context of variational algorithms for quantum chemistry? Secondly, in the context of NISQ devices, can we use these insights to build better ansätze, with reduced circuit depth and more efficient resources?
