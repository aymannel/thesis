A central challenge in computational quantum chemistry is the accurate simulation of fermionic systems. At the heart of these calculations lies the need to solve the Schrödinger equation to determine the many-electron wavefunction. Finding an exact solution to this problem scales exponentially with the number of electrons. Classical computers struggle to store the increasingly large wavefunctions making this problem classically intractable in many cases. In contrast, gate-based quantum computing presents a promising solution, offering the potential to represent electronic wavefunctions with polynomially scaling resources [4]. In other words, quantum computers are a natural tool of choice for simulating processes that are inherently quantum [2].

In the last two decades many advancements in quantum computing have been made in both hardware and software bringing us closer to being able to simulate molecular systems. However, we remain in the so-called Noisy Intermediate Scale Quantum (NISQ) era, and face challenges such as poor qubit fidelity, low qubit connectivity and limited coherence times. In particular, one class of quantum algorithms that has risen to fame is known as the Variational Quantum Eigensolver (VQE). This algorithm exploits the variational principle giving rise to a hybrid quantum-classical approach for finding the lowest eigenvale of the molecular Hamiltonian. The VQE algorithm involves preparing a trial wavefunction known as the ansatze, calculating its energy, then classically optimising the parameters of the wavefunction. This process is repeated until the energy converges to the optimal approximation for the groundstate [3]. \cite{Cowtan2019}
