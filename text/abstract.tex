A central challenge in computational quantum chemistry is the accurate simulation of fermionic systems. At the heart of these calculations lies the need to solve the Schrödinger equation to determine the many-electron wavefunction. The exact solution to this problem scales exponentially with the number of electrons, making it computationally intractable for large or strongly-correlated systems on classical computers, which cannot efficiently store the increasingly large wavefunctions \cite{Szalay2011}. In contrast, gate-based quantum computing offers the potential to represent electronic wavefunctions with polynomially scaling resources using quantum algorithms for chemical simulations \cite{Kassal2011}. In other words, quantum computers are a natural tool of choice for simulating processes that are inherently quantum \cite{Yeung2020}.

In the last two decades, significant advancements in quantum computing hardware and software have brought us closer to the ability to simulate molecular systems. Despite these advancements, we are still in the Noisy Intermediate Scale Quantum (NISQ) era \cite{Preskill2018}, characterised by challenges such as poor qubit fidelity, low qubit connectivity, and limited coherence times. The NISQ era represents a transitional phase in quantum computing, where quantum devices are not yet error-corrected but can perform computations beyond the capabilities of classical computers. Overcoming the limitations of the NISQ era is crucial for realising the full potential of quantum computing in various fields, including chemistry and materials science.

In this thesis, we focus on the Unitary Product State (UPS) ansatz which we use to represent fermionic wavefunctions on a quantum computer \cite{Burton2023} via the Variational Quantum Eigensolver (VQE) algorithm \cite{Wecker2015}. Specifically, we study the excitation operators used to prepare UPS ansätze using the ZX calculus, a diagrammatic language for reasoning about quantum processes \cite{Coecke2011}. The VQE algorithm is used to estimate the ground state energy of a molecular Hamiltonian by preparing a trial wavefunction, calculating its energy expectation value on a quantum device, and then optimising the wavefunction parameters classically until the energy converges to the best approximation of the ground state energy \cite{McClean2016}. It is recognised as a leading algorithm for quantum simulation on NISQ devices due to its reduced resource requirements in terms of qubit count and coherence time \cite{Kirby2020}.

We build on the work of Yeung \cite{Yeung2020} and Cowtan \textit{et al} \cite{Cowtan2020} on Pauli gadgets and Yordanov \textit{et al} \cite{Yordanov2020} and Kornell \textit{et al} \cite{Kornell2023} on fermionic excitation operators. Our research focuses on two main questions: Can we use the ZX calculus to gain insights into the structure of ansätze within VQE algorithms for quantum chemistry? Secondly, in the context of NISQ devices, can we use these insights to develop better ansätze with reduced circuit depth and more efficient resource usage? By attempting to reduce circuit depth, we address a major source of error in NISQ devices -- the noise of today's quantum hardware \cite{Cowtan2020}.

In Chapter \ref{background}, we introduce the theoretical foundation necessary for simulating molecules on quantum computers. In Chapter \ref{zx-calculus}, we present the ZX calculus, its generators, and its rewrite rules. Chapter \ref{pauli-gadgets} introduces Pauli gadgets, the building blocks for fermionic ansätze. In Chapter \ref{controlled-rotations}, we develop a representation for controlled rotations using the ZX calculus, applying this to replicate the work of Yordanov \textit{et al.} \cite{Yordanov2020} and Kornell \textit{et al.} \cite{Kornell2023} using the ZX calculus. In \textbf{Chapter \ref{excitation-operators}}, we combine the concepts discussed in the previous chapters to study the excitation operators used in the UPS ansatz, integrating the research ideas presented in this thesis. Finally, in Chapter \ref{zxfermion}, we introduce the ZxFermion software package, which we developed to facilitate the study of circuits of Pauli gadgets, and demonstrate how it can be used to replicate our findings.
