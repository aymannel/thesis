A central challenge in computational quantum chemistry is the accurate simulation of fermionic systems. At the heart of these calculations lies the need to solve the Schrödinger equation to determine the many-electron wavefunction. Since an exact solution to this problem scales exponentially with the number of electrons, and classical computers have no means by which to efficiently store the increasingly large wavefunctions, this problem becomes computationally intractable for large and strongly-correlated systems \cite{Szalay2011}. In contrast, gate-based quantum computing presents a promising solution, offering the potential to represent electronic wavefunctions with polynomially scaling resources using quantum algorithms for the simulation of chemical systems \cite{Kassal2011}. In other words, quantum computers are a natural tool of choice for simulating processes that are inherently quantum \cite{Yeung2020}.

In the last two decades, many advancements in quantum computing have been made in both hardware and software, bringing us closer to being able to simulate molecular systems. Despite these advancements, we remain in the so-called Noisy Intermediate Scale Quantum (NISQ) era \cite{Preskill2018}, characterised by challenges such as poor qubit fidelity, low qubit connectivity and limited coherence times \cite{Poulin2014}. The NISQ era represents a transitional phase in quantum computing, where quantum devices are not yet error-corrected but are still capable of performing computations beyond the reach of classical computers. Overcoming the limitations of the NISQ era is crucial for realising the full potential of quantum computing in various fields, including quantum chemistry and materials science.


In this thesis, we focus on the Unitary Product State (UPS) ansatz \cite{Burton2023}, implemented on quantum devices using the Variational Quantum Eigensolver (VQE) algorithm \cite{Wecker2015}. In particular, we are concerned with the study of the excitation operators used to prepare UPS ansätze representing fermionic wavefunctions. The VQE algorithm is a method used to estimate the ground state energy of a molecular Hamiltonian by preparing a trial wavefunction, calculating its energy expectation value on a quantum device, then optimising the wavefunction parameters classically until the energy converges to the best approximation for the ground state energy \cite{McClean2016}. It is recognised as a leading algorithm for quantum simulation on NISQ devices due to its reduced resource requirements in terms of qubit count and coherence time \cite{Kirby2020}.

We build on the work of Yeung \cite{Yeung2020} on Pauli gadgets, Yordanov \textit{et al} \cite{Yordanov2020} on fermionic excitation operators and Cowtwan \textit{et al} \cite{Cowtan2020}, concerning ourselves with two main questions: can we use the ZX calculus to gain insights into the structure of the UPS ansatz in the context of VQE algorithms for quantum chemistry? Secondly, in the context of NISQ devices, can we use these insights to build better ansätze with reduced circuit depth and more efficient resources? By attempting to reduce circuit depth, we are addressing the major source of error present in NISQ devices -- the noise of today's quantum hardware \cite{Cowtan2020}.

\begin{itemize}
    \item \textbf{Chapter \ref{background}} develops the mathematical foundation for simulating molecules on quantum computers.
    \item \textbf{Chapter \ref{zx-calculus}} introduces the generators of the ZX calculus and its rewrite rules.
    \item \textbf{Chapter \ref{pauli-gadgets}} introduces Pauli gadgets, the basic building blocks of fermionic ansätze, and their interaction with other quantum gates.
    \item \textbf{Chapter \ref{controlled-rotations}} explores controlled rotations in terms of phase polynomials.
    \item \textbf{Chapter \ref{excitation-operators}} applies the theory developed thus far to show how excitation operators can be expressed it terms of controlled rotations in the ZX calculus.
    \item \textbf{Chapter \ref{zxfermion}} introduces the software package ZxFermion that we built, demonstrating how it can be used to replicate the research done in this thesis.
\end{itemize}
