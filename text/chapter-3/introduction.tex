\chapter{\label{pauli-gadgets}Pauli Gadgets}

A Pauli string $P$ is defined as a tensor product of the members of the Pauli group $P \in \{I, Y, Z, X\}^{\otimes n}$, where each Pauli operator acts on a distinct qubit (indicated by the subscript). Note that the Pauli matrices are Hermitian operators, and therefore, so are Pauli strings. For convenience, we will drop the subscripts.

\includezxdiagramtext{chapter-3/pauli_string}{0.22}
{I_0 \,\, Z_1 \, X_2 = I \otimes Z \otimes X}

\textit{Stone's Theorem} \cite{Stone1932} states that a strongly continuous one parameter unitary group $U(\theta)$ is generated by a Hermitian operator, $P$.
\begin{equation*}
    U(\theta) = e^{i\theta P} = 1 + i\theta P +
    \frac{1}{2} [i\theta P]^2 +
    \frac{1}{6} [i\theta P]^3 + \dots
\end{equation*}

There is therefore a one-to-one correspondence between Hermitian operators and one parameter unitary groups \cite{Yeung2020}. The time evolution of a quantum mechanical system, described by the Hamiltonian $H$, is defined by the one parameter unitary group $e^{itH}$, whilst arbitrary rotation gates in the $Z$, $X$ and $Y$ bases are described by the one parameter unitary groups of the Pauli matrices $Z$, $X$ and $Y$.
\begin{equation*}
    R_Z(\theta) = e^{i\frac{\theta}{2} Z} \qquad
    R_Y(\theta) = e^{i\frac{\theta}{2} Y} \qquad
    R_X(\theta) = e^{i\frac{\theta}{2} X}
\end{equation*}

Phase gadgets are defied as the one parameter unitary groups of Pauli strings consisting of the $I$ and $Z$ matrices, $P \in \{I, Z\}^{\otimes n}$.
\begin{equation*}
    \Phi(\theta) = \text{exp} \left[
    i\frac{\theta}{2} (I \otimes Z \otimes Z \otimes \dots)
    \right]
\end{equation*}

In quantum circuit notation, phase gadgets are represented by a layer of CNOTs followed by a rotation in the $Z$ basis, followed by another layer of CNOTS.

\includezxdiagramtext{chapter-3/phase_gadget_expanded}{0.35}
{\text{exp} \left( i \frac{\theta}{2} Z \otimes Z \otimes Z \right)}

The first layer of CNOTs can be thought of as computing the parity of the input qubits by creating an entangled state. The rotation in the $Z$ basis then rotates the entangled state by $e^{i\theta/2}$ or $e^{-i\theta/2}$, depending on its parity. The final layer of CNOTs can be thought of as uncomputing the parity. Since phase gadgets correspond to diagonal unitary matrices in the $Z$ basis, they apply a global phase to a given state without changing the distribution of the observed state \cite{Yeung2020}.

There are many equivalent ways of expressing the same phase gadget in terms of CNOT gates and $Z$ rotation gates. Furthermore, the diagonal action of phase gadgets suggests that a more symmetrical structure exists for phase gadgets in the ZX calculus.

Indeed, we find that by iteratively applying the bialgebra rule, we can represent phase gadgets as follows.

\includeZxEqZxEq{chapter-3/ZIZ}{0.19}
{=\, \text{exp} \left( i \frac{\theta}{2} Z \otimes I \otimes Z \right)}
{chapter-3/ZZZ}{0.19}
{=\, \text{exp} \left( i \frac{\theta}{2} Z \otimes Z \otimes Z \right)}

By deforming our diagram \textit{d} and using the spider fusion \textit{f} (\ref{spider-fusion}), identity \textit{id} (\ref{identity}) and bialgebra \textit{ba} (\ref{bialgebra}) rules, we are able to show the correspondence between phase gadgets in normal quantum circuit form and in their form in the ZX calculus.

\includezxdiagram{chapter-3/phase_gadget_proof}{0.85}

It is then a simple matter of recursively applying this proof to phase gadgets in quantum circuit form to generalise it to arbitrary arity phase gadgets.

\includezxdiagram{chapter-3/phase_gadget_proof2}{1}
