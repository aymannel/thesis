\chapter{Pauli Gadgets}%
\label{pauli-gadgets}

Pauli gadgets are a class of unitary transformations that can be used to naturally represent the excitation operators used in UPS ansätze (Section \ref{ups-ansatz}). In this chapter, we introduce Pauli gadgets and their representation in the ZX calculus. We then develop a method for deriving a set of commutation relations that describes the interaction of Pauli gadgets with the Pauli and the Clifford gates. In Chapter \ref{excitation-operators}, we use Pauli gadgets to construct one-body and paired two-body excitation operators before using the commutation relations derived here in a number of derivations.

A Pauli string $P$ is defined as a tensor product of Pauli matrices $\{I, X, Y, Z\}^{\otimes n}$, where $n$ is the number of qubits in the system, and each Pauli gate acts on a distinct qubit. Thus $Z \otimes X$ represents the Pauli $Z$ and $X$ gates acting on the first and second qubits respectively. \textit{Stone's Theorem} \cite{Stone1932} states that a strongly-continuous one parameter unitary group $U(\theta) = \text{exp} \left(- i \frac{\theta}{2} H \right)$ is generated by the Hermitian operator $H$. Since Pauli strings are Hermitian, we can define Pauli gadgets as the one parameter unitary groups generated by the Pauli strings.

\begin{figure}[H]
    \centering
    \begin{gather*}
        \Phi_1(\theta) = \text{exp}\left(- i \frac{\theta}{2} Z \otimes I \otimes Z \right) \qquad
        \Phi_2(\theta) = \text{exp}\left(- i \frac{\theta}{2} Y \otimes Z \otimes X \right)%
    \end{gather*}
    \caption{Two examples of Pauli gadgets.}
\end{figure}
