\chapter{Pauli Gadgets}%
\label{pauli-gadgets}

Pauli gadgets are a class of unitary transformations that are implemented on quantum computers in terms of CNOT gates, $Z$ rotations and an associated change of basis. As we will see in Chapter \ref{excitation-operators}, Pauli gadgets are a natural way of reasoning about the excitation operators used to construct UPS ansätze.

A Pauli string $P$ is defined as a tensor product of Pauli matrices $P \in \{I, X, Y, Z\}^{\otimes n}$, where $n$ is the number of qubits in the system, and each Pauli gate acts on a distinct qubit. Thus $Z \otimes X$ represents the Pauli $Z$ and $X$ gates acting on the first and second qubits respectively.

\textit{Stone's Theorem} \cite{Stone1932} states that a strongly-continuous one parameter unitary group $U(\theta) = \text{exp} \left(- i \frac{\theta}{2} H \right)$ is generated by the Hermitian operator $H$. The Pauli matrices, and consequently Pauli strings, are Hermitian. We use the one-to-one correspondence between Hermitian operators and one parameter unitary groups to define Pauli gadgets as the one parameter unitary groups generated by Pauli strings.
\begin{gather*}
    \Phi_1(\theta) = \text{exp}\left(- i \frac{\theta}{2} Z \otimes I \otimes Z \right) \qquad
    \Phi_2(\theta) = \text{exp}\left(- i \frac{\theta}{2} Y \otimes Z \otimes X \right)%
\end{gather*}
