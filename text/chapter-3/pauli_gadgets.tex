\section{Pauli Gadgets}%
\label{pauli-gadgets-section}

Pauli gadgets are defined as the one parameter unitary groups of Pauli strings in the set $\{I, X, Y, Z\}^{\otimes n}$. Recognising that Pauli gadgets are simply phase gadgets associated with a change of basis, we can construct Pauli gadgets by conjugating the legs of a phase gadget with the appropriate Clifford gates, as described in Section \ref{clifford-conjugation}. Whilst phase gadgets alone cannot alter the distribution of the observed state, Pauli gadgets can \cite{Yeung2020}, and in Chapter \ref{excitation-operators}, we use Pauli gadgets to construct the one-body and two-body excitation operators used to in the UPS ansatz (\ref{ups-ansatz}).

\includezxdiagramtext{chapter-3/pauli_gadget}{0.6}{
\text{exp} \left( - i \frac{\theta}{2}
Y \otimes Z \otimes X \right)}

Pauli gadgets come equipped with a similar set of rules to phase gadgets that describe their interactions with other gadgets. For instance, adjacent Pauli gadgets with \textit{matching legs} fuse, and their phases add modulo $2\pi$.

\begin{figure}[H]
    \centering
    \includezxdiagram{chapter-3/pauli_gadget_fusion}{0.6}
    \caption{Pauli gadget fusion rule.}
    \label{pauli-gadget-fusion}
\end{figure}

Similar to the phase gadget commutation rule (\ref{phase-gadget-commutation}), we have that adjacent Pauli gadgets with \textit{no mismatching legs} commute.

\begin{figure}[H]
    \centering
    \includezxdiagram{chapter-3/pauli_gadget_commutation}{0.7}
    \caption{Pauli gadget commutation rule.}
    \label{pauli-gadget-commutation}
\end{figure}

Single-legged Pauli gadgets correspond to rotations in their respective basis.

\vspace{5pt}
\includezxdiagram{chapter-3/pauli_gadget_single_leg}{0.8}

