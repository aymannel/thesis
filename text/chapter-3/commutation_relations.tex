\section{Commutation Relations}%
\label{commutation-relations}

In this section we will develop a set of \textit{commutation relations} describing the interaction of Pauli gadgets with the Clifford gates. Diagrammatically, we interpret this as what happens to a Pauli gadget when a Clifford is pushed through it.

The Clifford group is the set of transformations that normalise the Pauli group. Therefore, conjugating a Pauli string by a member of the Clifford group $C^\dagger P \, C$ is closed in the set of Pauli strings, $\{I, X, Y, Z\}^{\otimes n}$, where $n$ is the number of qubits the Clifford gate acts on. Similarly, conjugating a Pauli gadget $\Phi(\theta)$ by a member of the Clifford group always yields some other Pauli gadget $\Phi'(\theta) = C^\dagger \Phi(\theta) C$. For instance, when $C = \text{CNOT}_{1, 2}$ and $\Phi(\theta) = \text{exp} \left[ i\frac{\theta}{2} Z \otimes Z \right]$, we can interpret the conjugation diagrammatically as the phase gadget decomposition result ($\ref{phase-gadget-decomposition}$).

\includezxdiagram{chapter-3/decomp}{0.55}

Recalling that the members of the Clifford group are unitary transformations, $C^{-1} = C^\dagger$, we can define \textit{commutation relation} more formally as $\Phi(\theta) C = C \, \Phi'(\theta)$.

\includezxdiagram{chapter-3/commutation}{0.55}

The CNOT gate is just one of many gates for which we can derive commutation relations. For example, consider the following Hadamard commutation relation.

\includezxdiagram{chapter-3/commutation_hadamard}{0.55}

%%%

\subsection{CNOT Commutation Relations}

The CNOT gate acts on two qubits ($n = 2$), therefore, we can form 16 Pauli gadgets from Pauli strings of the set of Pauli matrices, $\{I, X, Y, Z\}$ taken two at a time, with repetition. In other words, there are 16 unique commutation relations to derive to fully describe the interaction between the CNOT gate and Pauli gadgets.

\begin{figure}[H]
    \centering
    \includezxdiagram{chapter-3/CNOT_examples}{0.9}
    \caption{Two examples of the CNOT gate interacting with Pauli gadgets.}
\end{figure}

Whilst it is possible to derive each of these commutation relations using the Pauli gadget directly, it can be shown, through the Taylor expansion, that conjugating a Pauli gadget is equivalent to finding the one parameter unitary group of the conjugated Pauli string. Diagrammatically, this corresponds to pushing Pauli gates through some Clifford gate.

Let us illustrate the $\text{exp} \left[ i\frac{\theta}{2} Z \otimes Z \right] \text{CNOT}_{0, 1} = \text{CNOT}_{0, 1} \,\, \text{exp} \left[ i\frac{\theta}{2} I \otimes Z \right]$ commutation relation diagrammatically using the $Z \otimes Z$ Pauli string. We first push the bottom Pauli $Z$ gate through the CNOT target using the $\pi$ copy rule (\ref{pi-copy}), then, we push the top Pauli $Z$ gate through the CNOT control using the spider fusion rule (\ref{spider-fusion}), cancelling one of the copied Pauli $Z$ gates in the process.

\includezxdiagram{chapter-3/commutation_trick}{0.6}

The Pauli $Y$ gate can be expressed as a Pauli $X$ gate followed by a Pauli $Z$ gate, up to a global phase of $-i$ (CHECK). Below, we use this to illustrate the $\text{exp} \left[ i\frac{\theta}{2} Y \otimes X \right] \text{CNOT}_{0, 1} = \text{CNOT}_{0, 1} \,\, \text{exp} \left[ i\frac{\theta}{2} Y \otimes I \right]$ commutation relation.

\includezxdiagram{chapter-3/commutation_trick_YX}{1}

%%%

\subsection{CZ Commutation Relations}

Similarly, we can derive the 16 commutation relations of the CZ (controlled-Z) gate using the same method. Consider, for instance, the following example.

\includezxdiagram{chapter-3/commutation_cz}{1}

\subsection{Pauli Commutation Relations}

The Pauli matrices commute through Pauli strings by applying a global phase but leaving the Pauli string unchanged. Due to the one-to-one correspondence between Pauli strings and their corresponding Pauli gadget, we can show that the Paulis commute through Pauli gadgets by modifying their phase.

\includezxdiagram{chapter-3/commutation_paulis}{1}

As pointed out by Richie Yeung, a more efficient method for identifying these commutation relations is instead to construct a Clifford tableau as in Winderl \textit{et al} \cite{Yeung2023}. At the time, we were unaware of Clifford tableaus, however, using this method we were able to derive all 16 commutation relations (Appendix \ref{cnot_commutations}), demonstrating its efficacy. In Chapter \ref{zxfermion}, we choose to implement these commutation relations in our software package ZxFermion using Stim's (Clifford) \lstinline{Tableau} class.
