\section{Commutation Relations}%
\label{commutation-relations}

A pair of Pauli gadgets commute when their Hamiltonians commute. That is, a pair of Pauli gadgets commute when the Pauli strings that they are defined by also commute. Diagrammatically, we have that a pair of Pauli gadgets commute when they \textit{mismatch on an even number of legs} \cite{Yeung2020}. Let us demonstrate this by first considering the three possible mismatching pairs of Pauli gadget legs.

\includezxdiagram{chapter-3/mismatches}{1}

Starting with the mismatching $Z$/$X$ pair and using the bialgebra (\ref{bialgebra}), Hadamard commutation (\ref{hadamard-commutation}) and spider fusion (\ref{spider-fusion}) rules, we can show that commuting the gadgets' legs introduces a Hadamard between the bodies of the gadgets \cite{Yeung2020}.

\includezxdiagram{chapter-3/mismatch_proof}{1}

Similarly, for the $X$/$Y$ and $Y$/$Z$ mismatching pairs, we have the following.

\includezxdiagram{chapter-3/mismatch_proof2}{0.7}

Therefore, we can show that two Pauli gadgets with an even number of mismatching legs commute by first commuting all of their legs, then using the Hopf rule (\ref{hopf}) to remove the wires between them \cite{Yeung2020}.

\includezxdiagram{chapter-3/mismatch_proof3}{0.8}

\subsection{Clifford Commutation Relations}%
\label{clifford-commutation-relations}

We will now develop a set of \textit{commutation relations} describing the interaction of Pauli gadgets with members of the Clifford group. Diagrammatically, we interpret this as `what happens when a Clifford gate is pushed through a Pauli gadget'.

By definition, conjugating a Pauli string by a member of the Clifford group, $C^\dagger P \, C$, is closed in the set of Pauli strings, $\{I, X, Y, Z\}^{\otimes n}$, where $n$ is the number of qubits that the Pauli string acts on. Similarly, conjugating a Pauli gadget $\Phi(\theta)$ by a member of the Clifford group always yields another Pauli gadget $\Phi'(\theta) = C^\dagger \Phi(\theta) C$.

Taking $C = \text{CNOT}_{1, 2}$, we can interpret the conjugation of a \textit{phase gadget} diagrammatically (diagram on left) as the phase gadget decomposition result (\ref{phase-gadget-decomposition}). Recalling that the members of the Clifford group are unitary transformations, $C^{-1} = C^\dagger$, we define \textit{commutation relation} as $\Phi(\theta) C = C \, \Phi'(\theta)$ (diagram on right).

\includezxdiagram{chapter-3/commutation}{1}

Since the CNOT and CZ gates act on two qubits ($n=2$), we can form 16 unique Pauli gadgets from the set of Pauli strings $\{I, X, Y, Z\}^{\otimes 2}$. Consequently, we must derive 16 commutation relations to fully describe the interaction of Pauli gadgets with the CNOT and CZ gates.

% \begin{figure}[H]
%     \centering
%     \includezxdiagram{chapter-3/CNOT_examples}{0.9}
%     \caption{Example commutation relations. See Figures \ref{cnot-commutations} and \ref{cz-commutations} for more.}
% \end{figure}

Whilst it is possible to derive each of these commutation relations directly, it can be shown, through the relevant Taylor expansion, that conjugating a Pauli gadget is equivalent to finding the one parameter unitary group of the corresponding conjugated Pauli string. In other words, identifying how a Pauli string interacts with Clifford gates tells us how the corresponding Pauli gadget behaves.

Let us illustrate this with an example. Using the $Z \otimes Z$ Pauli string, we show that the $\text{CNOT}_{0, 1}$ gate commutes through the $\text{exp} \left[ - i\frac{\theta}{2} \left( Z \otimes Z \right) \right]$ gadget to give the $\text{exp} \left[ - i\frac{\theta}{2} \left( I \otimes Z \right) \right]$ gadget. We first push the bottom Pauli $Z$ gate through the CNOT target using the $\pi$ copy rule (\ref{pi-copy}), then, we push the top Pauli $Z$ gate through the CNOT control using the spider fusion rule (\ref{spider-fusion}), cancelling one of the copied Pauli $Z$ gates in the process.

\includezxdiagram{chapter-3/commutation_trick}{1}

The Pauli $Y$ gate is obtained by conjugating the Pauli $Z$ gate with Clifford gates (\ref{pauli-Y}). Using the $\pi$ copy (\ref{pi-copy}) and spider fusion (\ref{spider-fusion}) rules, we can show that $Y=XZ$ up to a global phase of $i$ and $Y=ZX$ up to a global phase of $-i$.

\includezxdiagram{chapter-3/clifford_conjugation}{1}

In the example below, one of the definitions of the Pauli $Y$ gate above to derive how the $\text{CNOT}$ gate interacts with the $\text{exp} \left[ - i\frac{\theta}{2} \left(Y \otimes X \right) \right]$ Pauli gadget by identifying how it gate interacts with the $Y \otimes X$ Pauli string.

\includezxdiagram{chapter-3/commutation_trick_YX}{0.95}

Similarly, we can identify how single-qubit gates interact with Pauli gadgets by identifying how they interact with the Pauli gates. In the example below, we use the fact that the above definitions of the Pauli $Y$ gate differ by a factor of $-1$ to show that commuting a Hadamard through a $Y$ leg flips the gadget's phase.

\includezxdiagram{chapter-3/single_qubit_clifford}{1}

Using the rules described in this section, we were able to identify how Pauli gadgets interact with the Clifford and Pauli gates, in agreement with \cite{Yeung2020} and \cite{Cowtan2019}. We have summarised these commutation relations below. In Chapter \ref{excitation-operators}, we demonstrate how these commutation relations can be used to prove several important results.

\begin{figure}[H]
    \centering
    \includezxdiagram{chapter-3/cnot_commutations}{0.9}
    \caption{CNOT commutation relations excluding repetitions.}
    \label{cnot-commutations}
\end{figure}

\begin{figure}[H]
    \centering
    \includezxdiagram{chapter-3/cz_commutations}{0.9}
    \caption{CZ commutation relations excluding repetitions.}
    \label{cz-commutations}
\end{figure}

\begin{figure}[H]
    \centering
    \includezxdiagram{chapter-3/pauli_commutations}{0.9}
    \caption{Pauli commutation relations.}
    \label{pauli-commutations}
\end{figure}

\begin{figure}[H]
    \centering
    \includezxdiagram{chapter-3/clifford_z_commutations}{0.9}
    \caption{Clifford $Z$ commutation relations.}
    \label{clifford-z-commutations}
\end{figure}

\begin{figure}[H]
    \centering
    \includezxdiagram{chapter-3/clifford_x_commutations}{0.9}
    \caption{Clifford $X$ commutation relations.}
    \label{clifford-x-commutations}
\end{figure}

\begin{figure}[H]
    \centering
    \includezxdiagram{chapter-3/clifford_h_commutations}{0.9}
    \caption{Hadamard commutation relations.}
    \label{clifford-h-commutations}
\end{figure}

% \subsection{Pauli Commutation Relations}

% The Pauli gates commute through Pauli strings by applying a global phase but leaving the Pauli string unchanged.

% \includezxdiagram{chapter-3/commutation_paulis}{1}

% \subsection{Clifford Tableaus}

% Richie Yeung pointed out late during the preparation of this thesis that a more efficient method of identifying such commutation relations is instead to construct a Clifford tableau as in Winderl \textit{et al} \cite{Yeung2023}. Nevertheless, the diagrammatic method introduced in this chapter allowed us to derive all 16 CNOT commutation relations (Appendix \ref{cnot_commutations}). In Chapter \ref{zxfermion}, we choose to implement these commutation relations in our software package ZxFermion using Stim's (Clifford) \lstinline{Tableau} class.
