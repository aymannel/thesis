\section{Commutation Relations}

In this section we will develop a set of commutation relations describing the interaction of Pauli gadgets with the Clifford gates. By \textit{commutation relation} we mean `what happens to a Pauli gadget when a Clifford is pushed through it'. The Clifford group is the set of quantum gates that normalise the Pauli group. That is, conjugating a member of the Pauli group $P$ by a Clifford gate $C$ results in another member of the Pauli group, $P' = C^\dagger P \, C$. Similarly, conjugating a Pauli gadget $\Phi(\theta)$ by a Clifford gate results in some other Pauli gadget $\Phi'(\theta)$. Let us consider the commutation relation $\Phi(\theta) C = C \, \Phi'(\theta)$. Assigning $C = \text{CNOT}_{0, 1}$ and $\Phi(\theta) = \text{exp} \left[ i\frac{\theta}{2} Z \otimes Z \right]$, we have the following commutation relation.

\includezxdiagram{chapter-4/CNOT_ZZ_IZ}{0.38}

Rearranging the commutation relation, we have that $\Phi'(\theta) = C^\dagger \Phi(\theta) C$. In other words, identifying the resulting Pauli gadget $\Phi'(\theta)$ amounts to conjugating $\Phi(\theta)$ with CNOT gates. Recall that diagrammatically, this is exactly what we do when we decompose phase gadgets using the bialgebra rule (\ref{bialgebra}).

\includezxdiagram{chapter-4/decomp}{0.4}

Similar commutation relations describe the interaction of CNOT gates with the various Pauli gadgets. For instance, consider the following commutation relations.


\includezxdiagram{chapter-4/CNOT_examples}{0.9}

There are 16 possible permutations with repetition of the set of Pauli matrices $\{I, X, Y, Z\}$ taken two at a time. That is, there are 16 unique Pauli gadgets with which a CNOT gate can interact. Whilst it is possible to derive the commutation relation for each Pauli gadget using the ZX calculus, there exists a simplifying method. Recall that Pauli gadgets are defined as the one parameter unitary groups of some Pauli string $P \in \{I, Z, X, Y\}^{\otimes n}$. It can be shown, through the relevant Taylor expansion, that conjugating a Pauli gadget is equivalent to finding the one parameter unitary group of the conjugated Pauli string. In other words, if we know how a Pauli string interacts with the CNOT gate, we can know how the corresponding Pauli gadget does too.

Let us illustrate the $\text{exp} \left[ i\frac{\theta}{2} Z \otimes Z \right] \text{CNOT}_{0, 1} = \text{CNOT}_{0, 1} \,\, \text{exp} \left[ i\frac{\theta}{2} I \otimes Z \right]$ commutation relation diagrammatically using the $Z \otimes Z$ Pauli string. Looking at the diagram below, we first push the bottom Pauli $Z$ gate through the CNOT target using the $\pi$ copy rule (\ref{pi-copy}). We then push the top Pauli $Z$ gate through the CNOT control using the spider fusion rule (\ref{spider-fusion}), which cancels one of the copied Pauli $Z$ gates in the process and yields $I \otimes Z$.

\includezxdiagram{chapter-4/commutation_trick}{0.6}

The Pauli $Y$ gate can be expressed as a Pauli $X$ gate followed by a Pauli $Z$ gate, up to a global phase of $-i$ (CHECK). Below, we use this to illustrate the $\text{exp} \left[ i\frac{\theta}{2} Y \otimes X \right] \text{CNOT}_{0, 1} = \text{CNOT}_{0, 1} \,\, \text{exp} \left[ i\frac{\theta}{2} Y \otimes I \right]$ commutation relation.

\includezxdiagram{chapter-4/commutation_trick_YX}{1}


As pointed out by Richie Yeung, a more efficient method for identifying these commutation relations is instead to construct a Clifford tableau as in Winderl \textit{et al} \cite{Yeung2023}. At the time, we were unaware of Clifford tableaus, however, using this method we were able to derive all 16 commutation relations (Appendix \ref{cnot_commutations}), demonstrating its efficacy. In Chapter \ref{zxfermion}, we choose to implement these commutation relations in our software package ZxFermion using Stim's (Clifford) \lstinline{Tableau} class.
