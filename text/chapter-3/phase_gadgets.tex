\section{Phase Gadgets}
Missing: matrix representation

Phase gadgets are a special type of Pauli gadget formed from Pauli strings consisting of the Pauli $I$ and $Z$ matrices. A Pauli string $P$ is defined as a tensor product of Pauli matrices $P \in \{I, Y, Z, X\}^{\otimes n}$, where $n$ is the number of qubits in the system. each Pauli operator acts on a distinct qubit. Note that since the Pauli matrices are Hermitian, so are Pauli strings. Thus $Z \otimes Y \otimes X = Z \, YX$ means apply Pauli $Z$, $Y$ and $X$ to qubits 0, 1 and 2 respectively.

\textit{Stone's Theorem} \cite{Stone1932} states that a strongly continuous one parameter unitary group $U(\theta)$ is generated by a Hermitian operator, $P$.
\begin{equation*}
    U(\theta) = \text{exp}\left(i \frac{\theta}{2} P \right)
    \quad \text{where} \quad
    e^X = 1 + X + \frac{1}{2} X^2 + \frac{1}{6} X^3 + \dots
\end{equation*}

There is therefore a one-to-one correspondence between Hermitian operators and one parameter unitary groups \cite{Yeung2020}. The time evolution of a quantum mechanical system, described by the Hamiltonian $H$, is defined by the one parameter unitary group $e^{itH}$, whilst arbitrary rotation gates in the $Z$, $X$ and $Y$ bases are described by the one parameter unitary groups of the Pauli matrices $Z$, $X$ and $Y$.
\begin{equation*}
    R_Z(\theta) = \text{exp}\left(i \frac{\theta}{2} Z \right) \qquad
    R_Y(\theta) = \text{exp}\left(i \frac{\theta}{2} Y \right) \qquad
    R_X(\theta) = \text{exp}\left(i \frac{\theta}{2} X \right)
\end{equation*}

Phase gadgets are defined as the one parameter unitary groups of Pauli strings consisting of the $I$ and $Z$ matrices, $P \in \{I, Z\}^{\otimes n}$. They correspond to quantum circuits made up of a $Z$ rotation wedged between two CNOT layers.

\includezxdiagramtext{chapter-3/phase_gadget_expanded}{0.28}
{\text{exp} \left( i \frac{\theta}{2} Z \otimes Z \otimes Z \right)}

The first CNOT layer can be thought of as computing the parity of the input state by entangling the qubits. The $Z$ rotation then rotates the entangled state by $e^{i\theta/2}$ or $e^{-i\theta/2}$, depending on the parity of the state, and the final CNOT layer uncomputes the parity. Phase gadgets necessarily correspond to diagonal unitary matrices in the $Z$ basis since they apply a global phase to a given state without changing the distribution of the observed state \cite{Yeung2020}. This diagonal action suggests that a symmetric ZX diagram exists for phase gadgets, as is indeed the case.

\includezxdiagramtext{chapter-3/phase_gadget_expanded_to_zx}{0.5}
{\text{exp} \left( i \frac{\theta}{2} Z \otimes Z \otimes Z \right)}

% \includeZxEqZxEq
% {chapter-3/ZIZ}{0.18}
% {=\, \text{exp} \left( i \frac{\theta}{2} Z \otimes I \otimes Z \right)}
% {chapter-3/ZZZ}{0.18}
% {=\, \text{exp} \left( i \frac{\theta}{2} Z \otimes Z \otimes Z \right)}

By deforming \textit{d} our phase gadget in quantum circuit notation and using the identity \textit{id} (\ref{identity}), spider fusion \textit{f} (\ref{spider-fusion}) and bialgebra \textit{ba} (\ref{bialgebra}) rules, we are able to show the correspondence with its form in the ZX calculus.

\includezxdiagram{chapter-3/phase_gadget_proof}{0.75}

It is then a simple matter of recursively applying this proof to phase gadgets in quantum circuit notation to generalise to arbitrary arity.

\includezxdiagram{chapter-3/phase_gadget_proof2}{0.8}%
\label{phase-gadget-proof}

Not only is this representation intuitively self-transpose, but it comes equipped with various rules describing its interaction with other phase gadgets and quantum gates.

%%%

\subsection{Phase Gadget Identity Rule}%
\label{phase-gadget-identity}

Phase gadgets with an angle $\theta = 0$ can be shown to be equivalent to identity using the state copy (\ref{state-copy}), spider fusion (\ref{spider-fusion}) and identity removal (\ref{identity}) rules.

\includezxdiagram{chapter-3/phase_gadget_identity}{0.8}

%%%

\subsection{Phase Gadget Fusion Rule}%
\label{phase-gadget-fusion}

Any two adjacent phase gadgets formed from the same Pauli string fuse and their phases add. This is achieved using the spider fusion rule (\ref{spider-fusion}) and the bialgebra rule (\ref{bialgebra}). See Appendix  \ref{appendix-phase-gadget-fusion} for the intermediate steps marked (*).

\includezxdiagram{chapter-3/phase_gadget_fusion}{0.5}

%%%

\subsection{Phase Gadget Commutation Rule}%
\label{phase-gadget-commutation}

Any two adjacent phase gadgets commute, as can be shown using the spider fusion rule \ref{spider-fusion} (fuse then unfuse the legs \ref{spider-fusion}).

\includezxdiagram{chapter-3/phase_gadget_commutation}{0.55}

%%%

\subsection{Single-Legged Phase Gadgets}%
\label{phase-gadget-single-leg}

Single-legged phase gadgets are equivalent to $Z$ rotations (see Appendix \ref{appendix-phase-gadget-single-leg}).

\includezxdiagram{chapter-3/phase_gadget_single_leg}{0.5}

%%%

\subsection{Phase Gadget Decomposition}%
\label{phase-gadget-decomposition}

There are many equivalent ways of decomposing a phase gadget into quantum circuit notation using the bialgebra rule (\ref{bialgebra}). More generally, we can show that it is possible to decompose a phase gadget such that it his a circuit depth of $\log_2(n)$ instead of $n$, where $n$ is the number of qubits.

\includezxdiagram{chapter-3/phase_gadget_decomposition}{0.85}

Clearly, it is then advantageous to manipulate circuits of phase gadgets using the ZX calculus, and to extract the corresponding quantum circuit only after the ZX diagram has been optimised.
