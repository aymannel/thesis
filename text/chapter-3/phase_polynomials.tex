\section{Phase Polynomials}%
\label{phase-polynomials}

Recalling that phase gadgets are diagonal in the computational $Z$ basis, we refer to a set of Pauli gadgets as diagonal when it consists only of phase gadgets \cite{Cowtan2020}. Such sets of phase gadgets are known as \textit{phase polynomials} and are themselves diagonal in the computational $Z$ basis \cite{Cowtan2019}. Phase polynomial synthesis refers to finding a quantum circuit that optimally implements some phase polynomial, for which there are several well-known algorithms \cite{Amy2013}, \cite{Amy2014}, \cite{Nam2018}.

As in Cowtan \textit{et al} \cite{Cowtan2020}, a set of Pauli gadgets $S$ can be simultaneously diagonalised by a Clifford subcircuit $C$ when all of the Pauli gadgets in the set commute. That is, by conjugating a set of commuting Pauli gadgets, we can re-express the circuit as some phase polynomial that can later be synthesised in some optimal way.

\begin{figure}[H]
    \centering
    \includezxdiagram{chapter-3/phase_polynomial}{0.95}
    \caption{Diagonalisation of a pair of commuting Pauli gadgets by $C$.}
\end{figure}

As we will see in Chapter \ref{excitation-operators}, the excitation operators used in VQE algorithms consist of commuting sets of Pauli gadgets. Therefore, the quantum circuits implementing such excitation operators can be diagonalised with some Clifford subcircuit $C$. As stated in Cowtan \textit{et al} \cite{Cowtan2020}, whilst diagonalisation may incur gate overhead, in practice, the reduction in circuit depth arising from synthesising the resulting phase polynomial usually more than makes up for the overhead.
