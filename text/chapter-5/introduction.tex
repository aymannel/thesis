\chapter{Excitation Operators}%
\label{excitation-operators}

In this chapter, we will use the theory that we have developed in Chapter \ref{pauli-gadgets} on Pauli gadgets and their commutation relations, and in Chapter \ref{controlled-rotations} on controlled rotations, to study the excitation operators used to construct Unitary-Product State ansätze.


\begin{itemize}[itemsep=-5pt]
    \item show example UCC ansatz in introduction using 'box' notation
    \item show commutation relations between different excitation operators
    \item introduce conjugation by some clifford to yield phase polynomial and suggest optimisation from \cite{Cowtan2020}
    \item \textbf{optimisations section} -- show balanced tree representation stuff "Circuit optimisation is typically carried out by pattern replacement: recognising a subcircuit of specific form and replacing it with an equivalent" \cite{Cowtan2019}
    \item "In principle, local rewriting of gate sequences is sufficient for any circuit optimisation3. However, in practice, good results often require manipulation of large-scale structures in the quantum circuit. Phase gadgets are one such macroscopic structure that is easy to identify within circuits, easy to synthesise back into a circuit, and have a useful algebra of interactions with one another." \cite{Cowtan2019}
    \item "Further, in the balanced tree form more of the CX gates are “exposed” to the rest of the circuit, and could potentially be eliminated by a later optimisation pass" \cite{Cowtan2019}
\end{itemize}
