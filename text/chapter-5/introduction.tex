\chapter{Excitation Operators}%
\label{excitation-operators}

In this chapter, we discuss how the excitation operators used to construct UPS ansätze can be implemented on a quantum computer. We then use the ZX calculus to represent these excitation operators as Pauli gadgets and discuss ways in which we can optimise their circuits with respect to circuit depth. We then demonstrate how these excitation operators can be expressed in terms of controlled rotations (see Chapter \ref{controlled-rotations}) following conjugation by some subcircuit. 

\begin{figure}[H]
    \centering
    \includezxdiagram{chapter-5/ucc}{0.7}
    \caption{Example UPS ansatz approximating some fermionic ground state.}
\end{figure}

As stated in Cowtan \textit{et al} \cite{Cowtan2020}, circuit optimisation amounts to pattern replacement, that is, recognising a subcircuit of a specific form and replacing it with an equivalent circuit that uses fewer quantum resources. Hence, by identifying the macroscopic structures representing these excitation operators, we facilitate the manipulation and optimisation of large-scale structures in the quantum circuit.
