\chapter{Excitation Operators}%
\label{excitation-operators}

In this chapter, we use the concepts discussed in the previous chapters to study one-body and paired two-body excitation operators using the ZX calculus. In Section \ref{implementing-excitation-operators}, we discuss the represention of the fermionic excitation operators in terms of Pauli gadgets as well as their specific circuit implementations. Then, in Section \ref{operator-commutations}, we discuss the several UPS ansätze generated by the DISCO-VQE algorithm, using the ZX calculus to attempt to rationalise their form. Finally, in Section \ref{operator-controlled-rotations}, we draw on the work done by Yordanov \textit{et al} \cite{Yordanov2020} and Kornell \textit{et al} \cite{Kornell2023}, to demonstrate how fermionic excitation operators can be expressed in terms of the controlled rotations discussed in Chapter \ref{controlled-rotations}.

\begin{figure}[H]
    \centering
    \includezxdiagram{chapter-5/ucc}{0.7}
    \caption{Example circuit implementation of the UPS ansatz.}
\end{figure}

% As stated in Cowtan \textit{et al} \cite{Cowtan2020}, circuit optimisation amounts to pattern replacement, that is, recognising a subcircuit of a specific form and replacing it with an equivalent circuit that uses fewer quantum resources. Hence, by identifying the macroscopic structures representing these excitation operators, we facilitate the manipulation and optimisation of large-scale structures in the quantum circuit.
