\chapter{Excitation Operators}%
\label{excitation-operators}

In this chapter, we will discuss how the excitation operators used to construct UCC ansätze can be implemented on a quantum computer. We will then use the ZX calculus to represent these excitation operators as Pauli gadgets (Section \ref{pauli-gadgets-section}) and discuss a number of circuit optimisation strategies in the context of the DISCO-VQE algorithm introduced in Burton \textit{et al} \cite{Burton2023}. Namely, we will demonstrate how these excitation operators can be expressed in terms of controlled rotations (Chapter \ref{controlled-rotations}), as well as discuss their representation as phase polynomials (Section \ref{phase-polynomials}) following diagonalisation by some Clifford subcircuit. 

\begin{figure}[H]
    \centering
    \includezxdiagram{chapter-5/ucc}{0.7}
    \caption{Example UCC ansatz approximating some fermionic ground state.}
\end{figure}

As stated in Cowtan \textit{et al} \cite{Cowtan2020}, circuit optimisation amounts to pattern replacement, that is, recognising a subcircuit of a specific form and replacing it with an equivalent circuit that uses fewer quantum resources. Hence, by identifying the macroscopic structures representing these excitation operators, we facilitate the manipulation of large-scale structures in the quantum circuit, as well as identify the means by which they interact with one another.
