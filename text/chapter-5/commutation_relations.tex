\section{Manipulating Circuits}

The \lstinline{GadgetCircuit} class comes equipped with the \lstinline{apply()} method that allows us to insert a quantum gate, and its Hermitian conjugate, into the circuit. This operation preserves the matrix corresponding to a given cricuit. The method takes the quantum gate to apply as its first parameter, whilst the \lstinline{start} and \lstinline{end} parameters designate the insertion positions.

\includejupyter{chapter-5/apply1}{chapter-5/apply1_zx}{0.4}

If no specific positions are specified, the insertion defaults to placing one quantum gate at the start, and its Hermitian conjugate at the end, of the circuit. The \lstinline{GadgetCircuit} class then manages the relevant commutation relations, ensuring the expected gadget outcome. Hence, we below, we observe the gadget that results upon pushing a CNOT gate through the \lstinline{Gadget("YZ", phase=1/2)} object.

\includejupyter{chapter-5/apply2}{chapter-5/apply2_zx}{0.4}

We could have chosen any Pauli gadget by simply instantiating the relevant \lstinline{Gadget()} object. Alternatively, we could have chosen to insert any Hermitian conjugate pair of gates. The commutation logic implements Stim's \lstinline{Tableau} class to identify the behaviour of a Pauli string with a given Clifford or Pauli gate.

\includejupyter{chapter-5/apply_cz}{chapter-5/apply_cz_zx}{0.45}

We can also use the \lstinline{GadgetCircuit} class to manipulate circuits consisting of multiple gadgets simultaneously. In the following circuit, we have instantiate a paired double excitation operator as a \lstinline{GadgetCircuit} object.

\includejupyter{chapter-5/double1}{chapter-5/double1_zx}{0.8}

Then, by conjugating the circuit with CNOTs, each with a different target qubit, we observe the following quantum circuit.

\includejupyter{chapter-5/double2}{chapter-5/double2_zx}{0.84}
