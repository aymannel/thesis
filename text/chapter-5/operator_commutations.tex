\section{Commuting Excitation Operators}%
\label{operator-commutations}

The DISCO-VQE algorithm, introduced in Section \ref{disco-vqe}, generates multiple UPS ansätze, each yielding the same energy expectation value but corresponding to a unique sequence of fermionic excitation operators. This indicates that the ansätze equivalently capture the correlation present in the molecular system of interest, implying potential redundancies in their representation. In this section, we discuss how to address this problem, aiming to rationalise the redundancies observed among the energy-equivalent ansätze. To achieve this goal, we use the Pauli gadget commutation rules outlined in Section \ref{commutation-relations} to identify several commuting excitation operators within the operator pool employed by the DISCO-VQE algorithm.

Recall from Section \ref{ups-ansatz} that the UPS ansatz allows us to parametrically explore the Hilbert space of possible states in search of a good approximation of the ground state. Provided we combine enough suitably-ordered fermionic excitation operators, the UPS ansatz can represent any vector in Hilbert space. One immediate question that arises from this is: what is the smallest finite sequence of $m$ fermionic excitation operators $\prod_i U_i(\theta_i)$, each parametrised by some angle $\theta_i$, that can equivalently traverse the Hilbert space as some other finite sequence of $n$ fermionic operators $\prod_j U_j(\theta_i)$? By answering this question, we anticipate being able to demonstrate the equivalence between some of the energy-equivalent ansätze by replacing one sequence of excitation operators in the ansatz with another sequence.

We begin by considering the fermionic excitation operators in the operator pool employed by the DISCO-VQE algorithm. We begin by noting that for a given ansatz, the ordering of qubits is defined such that spin-up and spin-down orbitals, which share the same spatial wavefunction, are adjacent \cite{Burton2023}, and that the fermionic excitation operators in the operator pool are chosen to conserve the spin symmetry of the initial state, as we discuss in Section \ref{disco-vqe}. Next, recall that in second quantisation, operators representing observables are constructed from the creation and annihilation operators presented in Section \ref{second-quantisation}. For instance, recalling the expression for the second-quantised Hamiltonian in Figure \ref{hamiltonian}, we see that the Hamiltonians of any two systems with the same number of spin orbitals differ only by the values of their matrix elements, $h_{ij}$ and $h_{ijkl}$. Similarly, we find that the same fermionic excitation operators can be used to construct ansätze for any molecular system, provided it has the same number of spin orbitals.

Recall that the Pauli gadgets comprising the one-body and paired two-body excitation operators mutually commute with each other by identifying that every pair of gadgets in these operators possesses two mismatching pairs of legs, as discussed in Section \ref{implementing-excitation-operators}. We then define \textit{trivially commuting} operators as commuting operators, that act on a disjoint set of qubits, and \textit{non-trivially commuting} operators as commuting operators that act on an intersecting set of qubits. Using this approach, we were able to successfully identify six previously-unknown pairs of non-trivially commuting excitation operators for linear H$_4$. Since the same fermionic excitation operators can be used to construct ansätze for any molecular system with the same number of spin orbitals, it follows that the operator commutation relations that we derived for linear H$_4$ are also true for square H$_4$ and trapezium H$_4$, as is indeed the case. More generally, these commutation relations apply to the fermionic excitation operators describing any fermionic system comprised of eight spin orbitals.

In contrast to traditional VQE algorithms, which optimise ansatz parameters continuously, the DISCO-VQE algorithm also incorporates discrete optimisation techniques. Specifically, it optimises the sequence of excitation operators through cyclic permutations, mutations, and pair swaps \cite{Burton2023}. In this context, we propose a modification to the DISCO-VQE algorithm to account for the additional non-trivially commuting excitation operators identified. We anticipate that by doing so, we might minimise redundancies among the expectation energy-equivalent ansätze, thereby decreasing the number of genuinely unique ansätze produced by the algorithm. We expect that this smaller set of ansätze may provide valuable insights into the correlations within the molecular system under study and potentially inform the development of more efficient optimisation algorithms and strategies.

