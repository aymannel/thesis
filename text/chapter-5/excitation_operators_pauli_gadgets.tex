\section{Excitation Operators as Pauli Gadgets}%
\label{excitation-operators-pauli-gadgets}

Using the phase gadget result (\ref{phase-gadget-result}), we can show that the quantum circuit implementing a one-body excitation operator corresponds to two commuting Pauli gadgets the ZX calculus.

\includezxdiagram{chapter-5/one_body3}{0.45}

One immediate advantage of representing the one-body excitation operator in this form is that we can easily show that these Pauli gadgets do indeed commute by recognising that they have two mismatching pairs of legs (see Section \ref{commutation-relations}). Another is that, from this intermediate form, we can resynthesise the quantum circuit using the balanced tree decomposition result (\ref{balanced-tree}), yielding a circuit depth of $2\text{log}_2(n)$ rather than $2(n-1)$.

\begin{figure}[H]
    \centering
    \includezxdiagram{chapter-5/balanced_tree}{0.65}
    \caption{Balanced tree decomposition of a one-body excitation operator.}
\end{figure}

Similarly, we can show that the two-body excitation operator corresponds to eight commuting Pauli gadgets, each of which can be optimally resynthesised using the balanced tree decomposition result (\ref{balanced-tree}).

\includezxdiagram{chapter-5/two_body1}{1}
