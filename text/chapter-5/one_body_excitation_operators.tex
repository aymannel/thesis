\section{One Body Excitation Operators}%
\label{one-body-excitation-operators}

Recall that a single fermionic excitation, from spin orbital $p$ to spin orbital $q$, can be expressed as $a^\dagger_q a_p$ in second quantisation.

By taking a linear combination of the excitation and de-excitations acting on qubits $p$ and $q$, we obtain the following anti-Hermitian fermionic one-body excitation operator.
\begin{equation*}
    \hat{\kappa}^q_p = a_q^\dagger a_p - a^\dagger_p a_q
\end{equation*}
Then, recalling the Jordan-Wigner transformation for the creation and annhilation operators,
\begin{equation*}
    \hat a_j^\dagger =
    \frac{1}{2} (X - iY) \otimes Z^\rightarrow_{j-1} \qquad
    \hat a_j =
    \frac{1}{2} (X + iY) \otimes Z^\rightarrow_{j-1}
\end{equation*}
We can show that the anti-Hermitian fermionic one-body excitation operator can be expressed in terms of quantum gates as follows.
\begin{equation*}
    F_p^q = \frac{i}{2} (Y_p X_q - X_p Y_q) \prod_{k=p+1}^{q-1} Z_k \\
\end{equation*}
By multiplying by $\theta$ and exponentiating, we obtain the corresponding unitary \textit{exponential one-body excitation operator}.
\begin{equation*}
    U^q_p (\theta) =
    \text{exp} \left( i
    \frac{\theta}{2} (Y_p X_q - X_p Y_q) \prod_{k=p+1}^{q-1} Z_k \right)
\end{equation*}
From here onwards, we will simply refer to the unitary exponential one-body excitation operator as the \textit{one-body excitation operator}.

We can express this one-body excitation operator as the following commuting exponential terms.
\begin{equation*}
\begin{gathered}
    U^q_p (\theta) =
    \text{exp} \left( i
    \frac{\theta}{2} (Y_p X_q - X_p Y_q) \prod_{k=p+1}^{q-1} Z_k \right) \\[1.5ex]
    %
    U^q_p (\theta) =
    \left( \text{exp} \left[
    i \frac{\theta}{2} Y_p X_q \prod_{k=p+1}^{q-1} Z_k \right] \right)
    %
    \left( \text{exp} \left[ -
    i \frac{\theta}{2} X_p Y_q \prod_{k=p+1}^{q-1} Z_k \right] \right)
\end{gathered}
\end{equation*}

The first exponential term can be implemented by the following quantum circuit.

\includezxdiagramtext{chapter-5/one_body1}{0.5}{
\left( \text{exp} \left[
i \frac{\theta}{2} Y_p X_q
\prod_{k=p+1}^{q-1} Z_k \right] \right)}

The second exponential term can be implemented by the following circuit.

\includezxdiagramtext{chapter-5/one_body2}{0.5}{
\left( \text{exp} \left[ -
i \frac{\theta}{2} X_p Y_q
\prod_{k=p+1}^{q-1} Z_k \right] \right)}

The left CNOT ladder can be thought of as calculating the parity of the fermionic state, whilst the right CNOT ladder construction uncomputes the parity.

By sequentially composing (\ref{composition}) these circuits, that is, taking their matrix product, we have implemented the one-body excitation operator between qubits $p$ and $q$. Expressing this operator as two Pauli gadgets, we obtain the following.

\includezxdiagramtext{chapter-5/one_body3}{0.5}{
\text{exp} \left( i
\frac{\theta}{2} (Y_p X_q - X_p Y_q)
\prod_{k=p+1}^{q-1} Z_k \right)}

\subsection{Excitation Operators as Controlled Rotations}

In this section, we will extend the work done by Yordanov \textit{et al} \cite{Yordanov2020}, looking at how excitation operators can be expressed in terms of a controlled rotation. To the knowledge of the author, this work has not yet been done in the ZX calculus. Let us take $U^1_0(\theta)$, $p = 0$ and $q = 1$, as a minimal example.

\includezxdiagram{chapter-5/minimal}{0.9}

By inserting two adjacent $\text{CNOT}_{0, 1}$ gates and using the CNOT commutation rules derived in Chapter \ref{pauli-gadgets}, we are able to show that the one-body excitation operator can be expressed in terms of a singly-controlled rotation rotation in the $Y$ basis.

\includezxdiagram{chapter-5/minimal_proof}{1}

Had we instead chosen to insert $\text{CNOT}_{1, 0}$ gates, we would have obtained the following controlled rotation, with the control on qubit 1.

\includezxdiagram{chapter-5/minimal2}{0.6}

Let us now look at the general case. Yordanov \textit{et al} \cite{Yordanov2020} show that a one-body excitation operator can be expressed as a controlled rotation in the $Y$ basis by conjugating with the following Cliffords. See Appendix \ref{appendix-one-body-general} for the intermediate steps.

\includezxdiagram{chapter-5/one_body_general}{1}

