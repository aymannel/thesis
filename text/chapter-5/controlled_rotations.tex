\section{Excitations as Controlled Rotations}%
\label{operator-controlled-rotations}

We began this research with the goal of identifying a generalised structure for the excitation operators used in the DISCO-VQE algorithm, in the ZX calculus. We anticipated that by doing so, we might discover novel ways of optimising UCC ansätze by developing a representation for these excitation operators that is independent of specific architectural constraints, as well as identifying rules describing their interactions.

This led us to the work done by Yordanov \textit{et al} \cite{Yordanov2020} and Kornell \textit{et al} \cite{Kornell2023}, which show that it is possible to reveal an underlying controlled rotation within each excitation operator by conjugating it with some subcircuit $S$. Hence, in this chapter, we replicate the results of Yordanov and Kornell in the ZX calculus using the results that we developed in previous chapters.

\begin{figure}[H]
    \centering
    \includezxdiagram{chapter-5/controlled_rotation}{0.75}
    \caption{Revealling the controlled rotation CR$(\theta)$ from the excitation operator $U(\theta)$.}
\end{figure}

Since there exist several efficient implementations of controlled rotations \cite{ZomorodiMoghadam2016}, \cite{Ye2006}, by rewriting the UCC ansatz in terms of controlled rotations, then substituting with a more efficient implementation, we can drastically reduce the circuit depth.

We begin by demonstrating the result in Yordanov \textit{et al} by using a minimal one-body excitation operator, implemented by the following quantum circuit.

\begin{figure}[H]
    \centering
    \includezxdiagram{chapter-5/minimal}{0.9}
    \caption{Quantum circuit (left) and in Pauli gadget form (right).}
    \label{minimal-one-body}
\end{figure}

We can show that by inserting two adjacent $\text{CNOT}_{0, 1}$ gates (self-inverse) into the circuit and using the CNOT commutation rules derived in Chapter \ref{pauli-gadgets}, the one-body excitation operator can be expressed in terms of a singly-controlled $Y$ rotation.

\includezxdiagram{chapter-5/minimal_proof}{1}

Had we instead chosen to insert two $\text{CNOT}_{1, 0}$ gates, we would have obtained the following controlled rotation, with the control on qubit 1.

\includezxdiagram{chapter-5/minimal2}{0.57}

Let us now look at the general one-body excitation operator discussed in Section \ref{excitation-operators-pauli-gadgets}. We begin by eliminating the $Z$ legs responsible for calculating the parity of the fermionic state, then, by expressing the single-legged Pauli gadget as a single-qubit rotation in the $Y$ basis and fusing it with the adjacent Pauli gadget, we obtain the following controlled $Y$ rotation.

\includezxdiagram{chapter-5/one_body_general_proof}{0.9}

\newpage
Now let us consider the two-body excitation operator acting on qubits $p$, $q$, $r$ and $s$. By conjugating the circuit with three CNOT gates as well as two Pauli $X$ gates, we obtain the following.

\includezxdiagram{chapter-5/two_body_proof}{1}

Then, by cancelling adjacent Clifford gates, we obtain the a phase polynomial conjugated by Cliffords. Recognising this result as the triply-controlled rotation derived in Chapter \ref{controlled-rotations}, we have demonstrated the result in Yordanov \textit{et al}.

\includezxdiagram{chapter-5/two_body2}{1}
