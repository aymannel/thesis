\section{Excitations as Controlled Rotations}%
\label{operator-controlled-rotations}

Recall that our objective is to establish a generalised representation for excitation operators using the ZX calculus. To achieve this, we draw upon the work of Yordanov \textit{et al} \cite{Yordanov2020} and Kornell \textit{et al} \cite{Kornell2023}, as discussed in this section. We replicate their findings within the ZX calculus framework, using the representation of higher-order controlled rotations developed in Chapter \ref{controlled-rotations} and the commutation relations derived in Chapter \ref{clifford-commutation-relations}. 

Fermionic excitation operators implemented as quantum circuits exhibit significant circuit depths due to the numerous CNOT gates constructions necessary to compute the parity of the fermionic state. Both Yordanov \textit{et al} and Kornell \textit{et al} propose that by rewriting the excitation operators in terms of controlled rotations, through conjugation with some subcircuit, and subsequently selecting a more efficient implementation for the controlled rotation, the circuit depth can be substantially decreased.

\begin{figure}[H]
    \centering
    \includezxdiagram{chapter-5/controlled_rotation}{0.7}
    \caption{Conjugating $U(\theta)$ by subcircuit $S$ to reveal a controlled rotation CR$(\theta)$.}
\end{figure}

To illustrate this, we begin by demonstrating the outcome of conjugating a minimal one-body excitation operator with CNOT gates using the ZX calculus.

\begin{figure}[H]
    \centering
    \includezxdiagram{chapter-5/minimal}{0.9}
    \caption{Quantum circuit (left) and in Pauli gadget form (right).}
    \label{minimal-one-body}
\end{figure}

By inserting two adjacent CNOT gates, with $p$ = control and $q$ = target, and using the CNOT commutation relations in Figure \ref{cnot-commutations}, we reveal two Pauli gadgets corresponding to a singly-controlled $Y$ rotation, as discussed in Section \ref{singly-controlled-rotations}.

\includezxdiagram{chapter-5/minimal_proof}{1}

Had we instead chosen $p$ = target and $q$ = control, we would have obtained the following singly-controlled $Y$ rotation, now controlled by the other qubit.

\includezxdiagram{chapter-5/minimal2}{0.57}

Let us now look at a general one-body excitation operator. We begin by conjugating with CNOT gates and a CZ gate (\ref{cz-definition}) using the CNOT commutation relations in Figure \ref{cnot-commutations} and CZ commutation relations in Figure \ref{cz-commutations}.

\includezxdiagram{chapter-5/one_body_general_proof1}{1}

Then, by fusing the single-qubit $Y$ rotation with the adjacent Pauli gadget, we identify the following singly-controlled $Y$ rotation. In this way, we have succesfully replicated the result in Yordanov \textit{et al} \cite{Yordanov2020} using the ZX calculus.

\includezxdiagram{chapter-5/one_body_general_proof2}{1}

Let us now consider the two-body excitation operator, represented as follows.

\includezxdiagram{chapter-5/two_body_proof1}{1}

We begin by conjugating the circuit with three CNOT gates using the CNOT commutation relations in Figure \ref{cnot-commutations}.

\includezxdiagram{chapter-5/two_body_proof2}{0.9}

Then, by cancelling the adjacent Clifford gates, we reveal a phase polynomial.

\includezxdiagram{chapter-5/two_body_proof3}{1}

Whilst the phase gadgets we obtain have the correct distribution of legs, their phases do not match those of a triply-controlled rotation. We correct this by conjugating the circuit with Pauli $X$ gates on the second and fourth qubits, using the Pauli $X$ commutation relations in Figure \ref{pauli-commutations}, ensuring that the phase polynomial exactly matches triply-controlled $Z$ rotation in Figure \ref{cccrz}. Note that by doing so, the controls on the second and fourth qubits are now controlled by the $\ket 0$ state, rather than the $\ket 1$ state.

\includezxdiagram{chapter-5/two_body_proof4}{1}

Finally, recognising that the target qubit (first qubit) is in the $Y$ basis through conjugation by Clifford gates, as discussed in Section \ref{clifford-conjugation}, we have, succesfully replicated the result in Yordanov \textit{et al} \cite{Yordanov2020} using the ZX calculus.

At the expense of introducing additional quantum gates, we have eliminated the resource-intensive CNOT ladder constructions associated with the two subcircuits in the one-body excitation operator and the eight subcircuits in the two-body excitation operator. By choosing efficient implementations of controlled $Y$ rotations, we can then offset the gate overhead introduced by the additional quantum gates.

Similarly, Kornell \textit{et al} \cite{Kornell2023} show that by conjugating the two-body excitation operator with the following subcircuit, we obtain the same triply-controlled $Y$ rotation. We were able to demonstrate this using the ZX calculus as follows.

\includezxdiagram{chapter-5/selinger1}{1}

In Chapter \ref{zxfermion}, we introduce a software package called ZxFermion that we developed to facilitate this type of derivation. Using ZxFermion, we were able to identify that it is possible to choose the control and target qubits of the triply-controlled rotation by conjugating with the following subcircuits.

\includezxdiagram{chapter-5/selinger2}{1}

\includezxdiagram{chapter-5/selinger3}{1}

\includezxdiagram{chapter-5/selinger4}{1}
