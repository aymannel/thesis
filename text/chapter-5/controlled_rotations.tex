\section{Excitations as Controlled Rotations}%
\label{operator-controlled-rotations}

In this section, we will extend the work done by Yordanov \textit{et al} \cite{Yordanov2020}, looking at how excitation operators can be expressed in terms of controlled rotations. To the knowledge of the author, this work has not yet been done in the ZX calculus. Taking $U^1_0(\theta)$, $p = 0$ and $q = 1$ as an example, we have the following.

\includezxdiagram{chapter-5/minimal}{0.9}

By inserting two adjacent $\text{CNOT}_{0, 1}$ gates (self-inverse) into the circuit and using the CNOT commutation rules derived in Chapter \ref{pauli-gadgets}, we are able to show that the one-body excitation operator can be expressed in terms of a singly-controlled rotation rotation in the $Y$ basis.

\includezxdiagram{chapter-5/minimal_proof}{1}

Had we instead chosen to insert two $\text{CNOT}_{1, 0}$ gates, we would have obtained the following controlled rotation, with the control on qubit 1.

\includezxdiagram{chapter-5/minimal2}{0.6}

Let us now look at the general case. Yordanov \textit{et al} \cite{Yordanov2020} show that a one-body excitation operator can be expressed as a controlled rotation in the $Y$ basis. See Appendix \ref{appendix-one-body-general} for the intermediate steps.

\includezxdiagram{chapter-5/one_body_general}{1}

Using the commutation relations derived in Chapter \ref{pauli-gadgets}, we can see that by conjugating the circuit with CNOT gates we obtain the following phase polynomial.

\includezxdiagram{chapter-5/two_body_proof}{1}

By cancelling adjacent change of basis gates, we can show that a two-body excitation operator corresponds to a triply-controlled rotation in the $Y$ basis (recalling the result for triply controlled rotations in Chapter \ref{controlled-rotations}), as in Yordanov \textit{et al}.

\includezxdiagram{chapter-5/two_body2}{1}
