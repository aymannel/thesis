\section{Excitations as Controlled Rotations}%
\label{operator-controlled-rotations}

Recall that our objective is to establish a generalised representation for excitation operators using the ZX calculus. To achieve this, we draw upon the work of Yordanov \textit{et al} \cite{Yordanov2020} and Kornell \textit{et al} \cite{Kornell2023} and replicate their findings within the ZX calculus framework, using the representation of higher-order controlled rotations developed in Chapter \ref{controlled-rotations} and the commutation relations derived in Chapter \ref{clifford-commutation-relations}. 

Fermionic excitation operators implemented as quantum circuits exhibit significant circuit depths due to the numerous CNOT gates required to compute the parity of the fermionic state. Both Yordanov \textit{et al} and Kornell \textit{et al} propose that by rewriting the excitation operators in terms of controlled rotations, through conjugation with some subcircuit, and subsequently selecting a more efficient implementation for the controlled rotation, the circuit depth can be substantially decreased.

\begin{figure}[H]
    \centering
    \includezxdiagram{chapter-5/controlled_rotation}{0.7}
    \caption{Conjugating $U(\theta)$ by subcircuit $S$ to reveal a controlled rotation CR$(\theta)$.}
\end{figure}

To illustrate this, we begin by demonstrating the outcome of conjugating a minimal one-body excitation operator with CNOT gates using the ZX calculus.

\begin{figure}[H]
    \centering
    \includezxdiagram{chapter-5/minimal}{0.9}
    \caption{Quantum circuit (left) and in Pauli gadget form (right).}
    \label{minimal-one-body}
\end{figure}

By inserting two adjacent CNOT gates, where $p$ = control and $q$ = target, and using the CNOT commutation relations in Figure \ref{cnot-commutations}, we reveal two Pauli gadgets.

\includezxdiagram{chapter-5/minimal_proof1}{0.9}

Recognising the single-legged Pauli gadget as a $Y$ rotation and fusing it with the two-legged Pauli gadget, yields the CR$_Y(\theta)$ gate introduced in Section \ref{singly-controlled-rotations}.

\includezxdiagram{chapter-5/minimal_proof2}{1}

Had we instead chosen $p$ = target and $q$ = control, we would have obtained the following singly-controlled $Y$ rotation, now controlled by the other qubit.

\includezxdiagram{chapter-5/minimal2}{0.55}

In the general case, we begin by conjugating with CNOT gates and a CZ gate (\ref{cz-definition}) using the commutation relations in Figures \ref{cnot-commutations} and \ref{cz-commutations}.

\includezxdiagram{chapter-5/one_body_general_proof1}{1}

As before, by fusing the resulting single-qubit $Y$ rotation with the adjacent Pauli gadget, we obtain the CR$_Y(\theta)$ gate. Therefore, we have succesfully replicated the result in Yordanov \textit{et al} using the ZX calculus.

\includezxdiagram{chapter-5/one_body_general_proof2}{1}

Let us now consider the following two-body excitation operator.

\includezxdiagram{chapter-5/two_body_proof1}{1}

We begin by conjugating the circuit with three CNOT gates, using Figure \ref{cnot-commutations}.

\includezxdiagram{chapter-5/two_body_proof2}{1}

Then, by cancelling the adjacent Clifford gates, we reveal a phase polynomial.

\includezxdiagram{chapter-5/two_body_proof3}{1}

Whilst the phase gadgets we obtain have the correct distribution of legs, their phases do not match those of the triply-controlled rotation shown in Figure \ref{cccrz}. We correct this by conjugating the second and fourth qubits with Pauli $X$ gates, using the commutation relations in Figure \ref{pauli-commutations}, then fuse them with the CNOT targets. Note that the controls conjugated by Pauli $X$ gates are now controlled by the $\ket 0$ state, rather than the $\ket 1$ state, as denoted by the white control nodes.

\includezxdiagram{chapter-5/two_body_proof4}{1}

Finally, recognising that the target (top) qubit is in the $Y$ basis, we have revealed a CCCR$_Y(\theta)$ gate and, therefore, succesfully replicated the result in Yordanov \textit{et al}. Note that above, we use `$\simeq$' since we haven't included the conjugation by the Clifford subcircuit in the left side of the equality. We have, however, implicitly included the Pauli $X$ gates by using white control nodes on the left.

Similarly, Kornell \textit{et al} \cite{Kornell2023} show that by conjugating the two-body excitation operator with the following subcircuit, we obtain the same triply-controlled $Y$ rotation. We were able to demonstrate this using the ZX calculus as follows.

\includezxdiagram{chapter-5/selinger1}{1}

In Chapter \ref{zxfermion}, we introduce a software package called ZxFermion that we developed to facilitate this type of derivation. Using ZxFermion, we were able to identify that it is possible to choose the control and target qubits of the triply-controlled rotation by conjugating with the following subcircuits.

\includezxdiagram{chapter-5/selinger2}{1}

\includezxdiagram{chapter-5/selinger3}{1}

\includezxdiagram{chapter-5/selinger4}{1}


At the expense of introducing additional quantum gates, we have eliminated the resource-intensive CNOT ladder constructions associated with the two subcircuits in the one-body excitation operator and the eight subcircuits in the two-body excitation operator. By choosing efficient implementations of controlled $Y$ rotations, we can then offset the gate overhead introduced by the additional quantum gates.

