\section{Implementing Excitation Operators}

In Section \ref{ups-ansatz}, we showed that a fermionic state can be represented by a finite sequence of $k$ parametrised one-body and two-body excitation operators acting on a single-Slater determinant reference state. After invoking the Trotter formula using one time step, we obtained the following equation for the UPS ansatz.
\begin{equation*}
    U(\bm\theta) = \prod_{m=1}^k U_m(\theta_m) \qquad
    U_m(\theta_m) = e^{\theta_m (\tau_m - \tau_m^\dagger)}
\end{equation*}
In order to implement the fermionic excitation operators $U_m(\theta_m)$ on a quantum computer, we must first represent the second-quantised creation and annhilation operators in terms of quantum gates. For a single-qubit system, we do this by taking the following linear combinations of the Pauli gates.
\begin{equation*}
    \hat a^\dagger = \ket 1 \bra 0 = \frac{1}{2} (X - iY) \qquad\qquad
    \hat a = \ket 0 \bra 1 = \frac{1}{2} (X + iY) 
\end{equation*}
Recalling that the creation and annhilation operators must preserve the fermionic anti-symmetry imposed by the Pauli principle, we modify these equations to account for the parity of the spin orbitals preceding the target spin orbital $j$ when dealing with many-qubit systems. We do this by introducing a string of Pauli $Z$ operators to compute the parity of the spin orbitals preceding spin orbital $j$.

\begin{figure}[H]
    \centering
    \begin{equation*}
        \hat a_j^+ = \frac{1}{2} (X - iY) \bigotimes_{k=1}^{j-1} Z_k \qquad\quad
        \hat a_j = \frac{1}{2} (X + iY) \bigotimes_{k=1}^{j-1} Z_k
    \end{equation*}
    \caption{The Jordan-Wigner transformation.}
\end{figure}

This mapping is known as the Jordan-Wigner transformation \cite{Seeley2020} and has the advantage of preserving a direct mapping between fermionic states in the occupation number representation and the qubit state vector. In other words, each qubit in our quantum circuit encodes the occupation number of each spin orbital in our fermionic state $\ket{f_{n-1} \dots f_{0}} \,\,\rightarrow\,\, \ket{q_{n-1} \dots q_{0}}$.

After applying the Jordan-Wigner transformation to the anti-Hermitian seecond-quantised operator $a_q^\dagger a_p - a_p^\dagger a_q$ and finding its corresponding unitary operator $U^q_p(\theta)$, we obtain the following expression in terms of the Pauli matrices.

% \begin{equation*}
%     U^q_p (\theta) =
%     \text{exp} \left( i
%     \frac{\theta}{2} (Y_p X_q - X_p Y_q) \prod_{k=p+1}^{q-1} Z_k \right)
% \end{equation*}

\begin{figure}[H]
    \centering
    \begin{equation*}
        U^q_p (\theta) =
        \text{exp} \left( i
        \frac{\theta}{2} (Y_p X_q - X_p Y_q) \prod_{k=p+1}^{q-1} Z_k \right)
    \end{equation*}
    \caption{One-body excitaiton operator expressed in terms of the Pauli matrices.}
    \label{one-body-excitation-operator}
\end{figure}

Similarly, by applying the Jordan-Wigner transformation to the two-body excitation operator derived from the second-quantised operator $a_r^\dagger a_s^\dagger a_q a_p - a_p^\dagger a_q^\dagger a_s a_r$, we obtain the following expression for $U^{rs}_{pq} (\theta)$.

% \begin{align*}
%     U^{rs}_{pq} (\theta) &= \text{exp} \left( i \frac{\theta}{8} (
%     X_p X_q Y_s X_r +
%     Y_p X_q Y_s Y_r +
%     X_p Y_q Y_s Y_r +
%     X_p X_q X_s Y_r - \right. \\
%     &\left. \hspace{1cm} 
%     Y_p X_q X_s X_r -
%     X_p Y_q X_s X_r -
%     Y_p Y_q Y_s X_r -
%     Y_p Y_q X_s Y_r )
%     \prod_{k=p+1}^{q-1} Z_k
%     \prod_{l=r+1}^{s-1} Z_l
%     \right)
% \end{align*}

\begin{figure}[H]
    \centering
    \begin{align*}
        U^{rs}_{pq} (\theta) &= \text{exp} \left( i \frac{\theta}{8} (
        X_p X_q Y_s X_r +
        Y_p X_q Y_s Y_r +
        X_p Y_q Y_s Y_r +
        X_p X_q X_s Y_r - \right. \\
        &\left. \hspace{1cm} 
        Y_p X_q X_s X_r -
        X_p Y_q X_s X_r -
        Y_p Y_q Y_s X_r -
        Y_p Y_q X_s Y_r )
        \prod_{k=p+1}^{q-1} Z_k
        \prod_{l=r+1}^{s-1} Z_l
        \right)
    \end{align*}
    \caption{Two-body excitation operator expressed in terms of the Pauli matrices.}
\end{figure}

Now that we have expressed the one-body and two-body excitation operators in terms of quantum gates, we discuss their specific implementations on a quantum computer. Starting with the one-body excitation operator, we can show that it can be expressed as the following product of commuting exponential terms.

\begin{figure}[H]
    \centering
    \begin{equation*}
        U^q_p (\theta) =
        \left( \text{exp} \left[
        i \frac{\theta}{2} Y_p X_q \prod_{k=p+1}^{q-1} Z_k \right] \right)
        \left( \text{exp} \left[ -
        i \frac{\theta}{2} X_p Y_q \prod_{k=p+1}^{q-1} Z_k \right] \right)
    \end{equation*}
    \caption{One-body excitation operator expressed as a product of exponential terms.}
\end{figure}
\vspace{-10pt}

Each exponential term can be implemented by a quantum circuit corresponding to a Pauli gadget in the ladder representation (\ref{pauli-gadgets-section}). Then, by sequentially composing (\ref{sequential-composition}) these circuits, that is taking their matrix product, we obtain the following quantum circuit implementing the one-body excitation operator.

% \includezxdiagramtext{chapter-5/one_body1}{0.45}{
% \left( \text{exp} \left[
% i \frac{\theta}{2} Y_p X_q
% \prod_{k=p+1}^{q-1} Z_k \right] \right)}

% And the second exponential term, as the following quantum circuit.

% \includezxdiagramtext{chapter-5/one_body2}{0.45}{
% \left( \text{exp} \left[ -
% i \frac{\theta}{2} X_p Y_q
% \prod_{k=p+1}^{q-1} Z_k \right] \right)}

% By sequentially composing these circuits, that is, taking their matrix product, we have implemented the one-body excitation operator acting on qubits $p$ and $q$.

\includezxdiagram{chapter-5/one_body4}{0.8}

As discussed in Section \ref{pauli-gadgets-section}, Pauli gadgets are simply phase gadgets associated with a change of basis. The phase gadget component is responsible for computing the parity of the spin orbitals between $p$ and $q$. Then, depending on this parity, it applies a phase to the state, thereby accounting for the fermionic antisymmetry imposed by the Pauli principle (Section \ref{many-electron-wavefunctions}). The one-body excitation operator, therefore, has the following representation in the ZX calculus.

\begin{figure}[H]
    \centering
    \includezxdiagram{chapter-5/one_body3}{0.45}
    \caption{Representation of a one-body excitation operator in the ZX calculus.}
\end{figure}

Similarly, the two-body excitation operator can be factorised into eight commuting exponential terms, and implemented by composing their subcircuits. The two-body excitation operator has the following representation in the ZX calculus.

\begin{figure}[H]
    \centering
    \includezxdiagram{chapter-5/two_body1}{1}
    \caption{Representation of a two-body excitation operator in the ZX calculus.}
\end{figure}

Although essential for accurately simulating fermionic systems, phase gadgets in the ladder representation are extremely resource-intensive and contribute significantly to the circuit depth. Hence, recalling the balanced tree representation (\ref{balanced-tree}), we can optimise the circuit such that it has a depth of $2\text{log}_2(n) + 1$ rather than $2n - 1$.

\begin{figure}[H]
    \centering
    \includezxdiagram{chapter-5/balanced_tree}{0.7}
    \caption{Balanced tree representation of a one-body excitation operator.}
\end{figure}
