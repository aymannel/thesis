\section{Implementing Excitation Operators}%
\label{implementing-excitation-operators}

In Section \ref{ups-ansatz}, we introduced the UPS ansatz, which can represent any state using a sufficiently large sequence of $k$ parametrised one-body and paired two-body excitation operators.
\begin{equation*}
    U(\bm\theta) = \prod_{m=1}^k U_m(\theta_m) \qquad
    U_m(\theta_m) = e^{\theta_m (\tau_m - \tau_m^\dagger)}
\end{equation*}
In order to implement the fermionic excitation operators $U_m(\theta_m)$ on a quantum computer, we must first represent the second-quantised creation and annhilation operators in terms of quantum gates. For a single-qubit system, we do this by taking the following linear combinations of the Pauli gates.
\begin{equation*}
    \hat a^\dagger = \ket 1 \bra 0 = \frac{1}{2} (X - iY) \qquad\qquad
    \hat a = \ket 0 \bra 1 = \frac{1}{2} (X + iY) 
\end{equation*}
Recalling that the creation and annhilation operators must preserve the fermionic anti-symmetry imposed by the Pauli principle, we modify these equations to account for the parity of the spin orbitals preceding the target spin orbital $j$ when dealing with many-qubit systems. We do this by introducing a string of Pauli $Z$ operators to compute the parity of the spin orbitals preceding spin orbital $j$.

\begin{figure}[H]
    \centering
    \begin{equation*}
        \hat a_j^+ = \frac{1}{2} (X - iY) \bigotimes_{k=1}^{j-1} Z_k \qquad\quad
        \hat a_j = \frac{1}{2} (X + iY) \bigotimes_{k=1}^{j-1} Z_k
    \end{equation*}
    \caption{The Jordan-Wigner transformation.}
    \label{jordan-wigner}
\end{figure}

This mapping is known as the Jordan-Wigner transformation \cite{Seeley2020} and has the advantage of preserving a direct mapping between fermionic states in the occupation number representation and the qubit state vector. In other words, each qubit in our quantum circuit directly encodes the occupation number of each spin orbital in the fermionic state: $\ket{f_{n-1}, f_{n-2}, \dots f_1, f_0} \,\,\rightarrow\,\, \ket{q_{n-1}, q_{n-2}, \dots q_1, q_0}$.

After applying the Jordan-Wigner transformation to the anti-Hermitian seecond-quantised operator $a_q^\dagger a_p - a_p^\dagger a_q$ and finding its corresponding unitary operator $U^q_p(\theta)$, we obtain the following expression in terms of the Pauli matrices.

% \begin{equation*}
%     U^q_p (\theta) =
%     \text{exp} \left( i
%     \frac{\theta}{2} (Y_p X_q - X_p Y_q) \prod_{k=p+1}^{q-1} Z_k \right)
% \end{equation*}

\begin{figure}[H]
    \centering
    \begin{equation*}
        U^q_p (\theta) =
        \text{exp} \left( i
        \frac{\theta}{2} (Y_p X_q - X_p Y_q) \prod_{k=p+1}^{q-1} Z_k \right)
    \end{equation*}
    \caption{One-body excitaiton operator expressed in terms of the Pauli matrices.}
    \label{one-body-excitation-operator}
\end{figure}

Similarly, by applying the Jordan-Wigner transformation to the two-body excitation operator derived from $a_r^\dagger a_s^\dagger a_q a_p - a_p^\dagger a_q^\dagger a_s a_r$, we obtain the following.

% \begin{align*}
%     U^{rs}_{pq} (\theta) &= \text{exp} \left( i \frac{\theta}{8} (
%     X_p X_q Y_s X_r +
%     Y_p X_q Y_s Y_r +
%     X_p Y_q Y_s Y_r +
%     X_p X_q X_s Y_r - \right. \\
%     &\left. \hspace{1cm} 
%     Y_p X_q X_s X_r -
%     X_p Y_q X_s X_r -
%     Y_p Y_q Y_s X_r -
%     Y_p Y_q X_s Y_r )
%     \prod_{k=p+1}^{q-1} Z_k
%     \prod_{l=r+1}^{s-1} Z_l
%     \right)
% \end{align*}

\begin{figure}[H]
    \centering
    \begin{align*}
        U^{rs}_{pq} (\theta) &= \text{exp} \left( i \frac{\theta}{8} (
        X_p X_q Y_s X_r +
        Y_p X_q Y_s Y_r +
        X_p Y_q Y_s Y_r +
        X_p X_q X_s Y_r - \right. \\
        &\left. \hspace{1cm} 
        Y_p X_q X_s X_r -
        X_p Y_q X_s X_r -
        Y_p Y_q Y_s X_r -
        Y_p Y_q X_s Y_r )
        \prod_{k=p+1}^{q-1} Z_k
        \prod_{l=r+1}^{s-1} Z_l
        \right)
    \end{align*}
    \caption{Two-body excitation operator expressed in terms of the Pauli matrices.}
\end{figure}

Now that we have expressed the one-body and two-body excitation operators in terms of quantum gates, we discuss their specific implementations on a quantum computer. Starting with the one-body excitation operator, we can show that it can be rewritten as the following product of commuting exponential terms.

\begin{figure}[H]
    \centering
    \begin{equation*}
        U^q_p (\theta) =
        \left( \text{exp} \left[
        i \frac{\theta}{2} Y_p X_q \prod_{k=p+1}^{q-1} Z_k \right] \right)
        \left( \text{exp} \left[ -
        i \frac{\theta}{2} X_p Y_q \prod_{k=p+1}^{q-1} Z_k \right] \right)
    \end{equation*}
    \caption{One-body excitation operator expressed as a product of exponential terms.}
\end{figure}
\vspace{-10pt}

Recognising these exponential terms as the one parameter unitary groups of two Pauli strings, as we discussed in Section \ref{pauli-gadgets-section}, we see that they each correspond to a Pauli gadget. Consequently, by sequentially composing these Pauli gadgets (taking their matrix product) we obtain the following diagram in the ZX calculus. 

\begin{figure}[H]
    \centering
    \includezxdiagram{chapter-5/one_body3}{0.45}
    \caption{Representation of a one-body excitation operator in the ZX calculus.}
\end{figure}

We are only interested in paired two-body excitation operators (Section \ref{disco-vqe}), which are constructed from the following four-legged Pauli gadgets. Each Pauli gadget can, therefore, be implemented by a constant number of CNOT gates \cite{Burton2023}.

\begin{figure}[H]
    \centering
    \includezxdiagram{chapter-5/two_body1}{1}
    \caption{Representation of a paired two-body excitation operator in the ZX calculus.}
\end{figure}

At this point, it is worth acknowledging the expressive power of the ZX calculus. Had we been using traditional circuit notation to represent the one-body excitation operator, we would have needed to choose a specific circuit implementation. Quantum computing literature usually discusses this operator in the ladder representation, which gives no indication of the many possible equivalent implementations.

\includezxdiagram{chapter-5/one_body4}{0.85}

The commonly-used ladder representation isn't even the most efficient circuit implementation of the Pauli gadgets. Recalling the balanced tree representation (\ref{balanced-tree}), we can implement operator with a depth of $2\text{log}_2(n) + 1$ rather than $2n - 1$.

\includezxdiagram{chapter-5/balanced_tree}{0.7}

Furthermore, the extensive set of rules, introduced in Chapter \ref{pauli-gadgets}, which describe the interactions of Pauli gadgets in the ZX calculus make this description far more powerful. For instance, we can easily verify that the Pauli gadgets comprising fermionic excitation operators mutually commute by realising that each pair of Pauli gadgets has two mismatching pairs of legs, as discussed in Section \ref{commutation-relations}.
