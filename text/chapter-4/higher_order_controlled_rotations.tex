\section{Higher-Order Controlled Rotations}

We can implement higher order controlled rotations by nesting singly-controlled rotations in CNOT gates \cite{Yordanov2020}. For instance, we implement a doubly-controlled $Z$ rotation using two singly-controlled $Z$ rotations with opposite phases as follows.

\includezxdiagram{chapter-4/CCRZ1}{0.65}

We can minimise the depth of the circuit implementing the CCR$_Z(\theta)$ gate by choosing specific decompositions of the CR$_Z(\theta)$ gate (left diagram) such that one pair of CNOT gates are adjacent and cancel each another (right diagram).

\begin{figure}[H]
    \centering
    \includezxdiagram{chapter-4/CCRZ2}{1}
    \caption{CCR$_Z(\theta)$ circuit decomposition (left). Cancelling CNOT gate pair (right).}
\end{figure}

For the purposes of this thesis, we are interested in the form of controlled rotations in terms of Pauli gadgets. By expressing all $Z$ rotations as phase gadgets (\ref{phase-gadget-single-leg}) and commuting the CNOT gates through them using the commutation rules in Figure \ref{cnot-commutations}, we can cancel the CNOT gates and obtain the following phase polynomial.

\begin{figure}[H]
    \centering
    \includezxdiagram{chapter-4/CCRZ3}{0.7}
    \caption{Representation of the CCR$_Z(\theta)$ gate in the ZX calculus.}
    \label{ccrz}
\end{figure}

More generally, we can build controlled rotations of arbitrary arity by recursively nesting controled rotations as above. Hence, as with the doubly-controlled rotation, we can construct triply-controlled rotations by nesting two doubly-controlled rotations. Below is one specific circuit implementation of a triply-controlled $Z$ rotation that maximises the number of cancelling CNOT gates.

\begin{figure}[H]
    \centering
    \includezxdiagram{chapter-4/CCCRZ}{1}
    \caption{CCCR$_Z(\theta)$ gate.}
\end{figure}

As before, we can show that the CCCR$_Z(\theta)$ gate corresponds to a phase polynomial, this time consisting of eight commuting phase gadgets.

\begin{figure}[H]
    \centering
    \includezxdiagram{chapter-4/CCCRZ_zx}{1}
    \caption{Representation of the CCCR$_Z(\theta)$ gate in the ZX calculus.}
    \label{cccrz}
\end{figure}

\subsection{Controlled Rotations in Different Bases}

Starting with a controlled $Z$ rotation and surrounding the target qubit with Hadamard gates (\ref{hadamard-definition}), we obtain the corresponding controlled $X$ rotation.

\begin{figure}[H]
    \centering
    \includezxdiagram{chapter-4/CCRX}{0.7}
    \caption{Representation of the CCR$_X(\theta)$ gate in the ZX calculus.}
\end{figure}

Similarly, by surrounding the target qubit of a controlled $Z$ rotation with the Clifford gates described in Section \ref{clifford-conjugation}, we obtain a controlled $Y$ rotation.

\begin{figure}[H]
    \centering
    \includezxdiagram{chapter-4/CCCRY}{1}
    \caption{Representation of the CCCR$_Y(\theta)$ gate in the ZX calculus.}
    \label{cccry}
\end{figure}
