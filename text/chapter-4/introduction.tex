\chapter{Controlled Rotations}%
\label{controlled-rotations}

In our attempt to identify a generalised representation for excitation operators using the ZX calculus, we came across the work done by Yordanov \textit{et al}, which shows that excitation operators can be rewritten in terms of controlled rotations by through conjugation by some subcircuit. Our interpretation of this result is that controlled rotations can be used to account for the Pauli antisymmetry of fermionic systems by conditionally applying a phase (rotation) depending on the parity of the state. In other words, the rotation is \textit{controlled} by the parity of the state.

In this chapter, we begin by discussing the well-established representation of singly-controlled rotations the ZX calculus. We then demonstrate how these singly-controlled rotations can be used to construct higher-order controlled rotations, developing a representation for them in terms of phase polynomials. This approach was undertaken independently and the conclusions drawn in this section constitute a substantial portion of the research conducted in this thesis, requiring the use of the Clifford commutation relations developed in Section \ref{clifford-commutation-relations}.
