\chapter{Controlled Rotations}%
\label{controlled-rotations}

Controlled rotations play an important part in UCC ansätze representing fermionic systems. They can be used to account for the fermionic antisymmetry observed in fermionic systems by applying a phase (rotation) depending on the parity of the state. In other words, the rotation is \textit{controlled} by the parity of the state.

In this chapter, we develop a representation for higher-order controlled rotations in terms of phase polynomials, aiming to replicate the results described by Yordanov \textit{et al} \cite{Yordanov2020}. We begin by discussing the well-established representation of singly-controlled rotations the ZX calculus. Next, we demonstrate how these can be used to construct higher-order controlled rotations and show how they can be expressed as a circuit of Pauli gadgets.

% MODIFY PARA AND UNCOMMENT
This approach was undertaken independently and the conclusions drawn in this section constitute a substantial portion of the research conducted in this thesis, requiring the use of the Clifford commutation relations developed in Section \ref{clifford-commutation-relations}. In Chapter \ref{excitation-operators}, we then use the representations of higher-order controlled rotations developed in this chapter to demonstrate the correspondence between excitation operators and controlled rotations.
