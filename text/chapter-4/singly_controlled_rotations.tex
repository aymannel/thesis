\section{Singly Controlled-Rotations}

Starting with the singly-controlled $Z$ rotation gate CR$_Z$, we will see that it can be expressed as a combination of phase gadgets, since it corresponds to a diagonal matrix in the computational $Z$ basis \cite{Yeung2020}.

\includezxdiagram{chapter-4/CRZ1}{0.68}

The CR$_Z$ gate applies a rotation to the target qubit if the control qubit is in the $\ket 1$ state, and applies no rotation, if the control qubit is in the $\ket 0$ state. This behaviour generalises to superpositions of states. Using the spider fusion (\ref{spider-fusion}) and bialgebra (\ref{bialgebra}) rules, we obtain the CR$_Z$ gate in quantum circuit notation as follows.

\includezxdiagram{chapter-4/CRZ2}{0.72}

To verify that this circuit does indeed correspond to a controlled rotation in the $Z$ basis, let us observe what happens when the control qubit is $\ket 0$. Using the spider fusion (\ref{spider-fusion}), state copy (\ref{state-copy}) and identity (\ref{identity}) rules, we see that,

\includezxdiagram{chapter-4/CRZ_zero}{1}

As expected, fusing the $Z$ rotations in the resulting diagram yields identity. Conversely, when the control qubit is $\ket 1$, we obtain a $Z$ rotation by $\theta$.

\includezxdiagram{chapter-4/CRZ_one}{1}

Using that the $Z$ rotation commutes through the phase gadget (\ref{spider-fusion}), and the phase gadget decomposition result (\ref{phase-gadget-decomposition}), we can decompose the CR$_Z$ gate into four equivalent quantum circuits.

\includezxdiagram{chapter-4/CRZ_decomp}{1}
