\section{Singly-Controlled-Rotations}%
\label{singly-controlled-rotations}

We can express a singly-controlled $Z$ rotation gate CR$_Z(\theta)$, controlled by the second qubit, as two phase gadgets. This aligns with our expectations since the CR$_Z(\theta)$ gate is diagonal in the computational $Z$ basis \cite{Yeung2020}.

\begin{figure}[H]
    \centering
    \includezxdiagram{chapter-4/CRZ1}{0.68}
    \caption{Representation of the CR$_Z(\theta)$ gate in the ZX calculus.}
    \label{crz}
\end{figure}

The CR$_Z(\theta)$ gate applies a rotation to the target qubit if the control qubit is in the $\ket 1$ state, and applies no rotation, if the control qubit is in the $\ket 0$ state. This behaviour generalises to superpositions of states. Using the spider fusion (\ref{spider-fusion}) and bialgebra (\ref{bialgebra}) rules, we obtain the following in quantum circuit notation.

\includezxdiagram{chapter-4/CRZ2}{0.75}

To verify that this circuit does indeed correspond to a controlled rotation in the $Z$ basis, let us observe what happens when the control qubit is in the $\ket 0$ state. Using the spider fusion (\ref{spider-fusion}), state copy (\ref{state-copy}) and identity (\ref{identity}) rules,

\includezxdiagram{chapter-4/CRZ_zero}{1}

As expected, the resulting diagram fuses to give identity. Conversely, when the control qubit is in the $\ket 1$ state, we obtain a $Z$ rotation by $\theta$.

\includezxdiagram{chapter-4/CRZ_one}{1}

Using the the phase gadget fusion rule (\ref{phase-gadget-fusion}) and the phase gadget decomposition result (\ref{phase-gadget-decomposition}), we can decompose the CR$_Z(\theta)$ gate into four equivalent circuits.

\includezxdiagram{chapter-4/CRZ_decomp}{0.95}
