\section{Singly Controlled-Rotations}

In this section we will diagrammatically derive a singly-controlled rotation. A controlled rotation with respect to the $Z$ eigenbasis applies a rotation to the target qubit if the control qubit is in the $\ket 1$ state, and applies no rotation, if the control qubit is in the $\ket 0$ state.

Let us start by considering the behaviour of the CNOT gate with the $Z$ eigenstates. Using the state copy (\ref{state-copy}), we can show when the control qubit is $\ket 0$, we obtain identity, whilst when $\ket 1$ we obtain the Pauli $X$ gate.

\includezxdiagram{chapter-4/cnot_zero_one}{1}

Next, using the $\pi$ copy rule (\ref{pi-copy}), we can show that conjugating a $Z$ rotation by Pauli $X$ gates, negates the phase of the rotation, $\alpha \rightarrow -\alpha$.

\includezxdiagram{chapter-4/phase_flip}{0.7}

Combining these, we can construct a singly controlled rotation in the $Z$ basis.

\includezxdiagram{chapter-4/CRZ}{0.45}

We can easily verify this by observing the behaviour of the $Z$ eigenstates.

\includezxdiagram{chapter-4/CRZ_zero}{1}

\includezxdiagram{chapter-4/CRZ_one}{1}

The resulting diagrams fuse to give identity or the Pauli $X$ gate, as expected.

\includezxdiagram{chapter-4/CRZ_fuse}{0.85}

Using the bialgebra rule (\ref{bialgebra}), we identify a two-legged phase gadget and a $Z$ rotation.

\includezxdiagram{chapter-4/CRZ_zx}{0.8}

Using that the $Z$ rotation commutes through the phase gadget (\ref{spider-fusion}), and the phase gadget decomposition rule (\ref{phase-gadget-decomposition}) , we obtain four equivalent diagrams. 

\includezxdiagram{chapter-4/CRZ_decomp}{1}

Deriving singly-controlled rotations in the $X$ and $Y$ bases amounts to conjugating the target qubit by the correct gates.

\begin{figure}[H]
    \centering
    \includezxdiagram{chapter-4/CRX_CRY}{0.75}
    \caption{Singly-controlled $X$ (left) and $Y$ (right) rotations.}
\end{figure}

