\section{Electronic Structure Theory}

\subsection{Electronic Structure Problem}
The main interest of electronic structure theory is finding approximate solutions to the non-relativistic time-independent Schrödinger equation.
\begin{equation*}
    H\ket\psi = E\ket\psi
\end{equation*}
It is an eigenvalue equation of the full molecular Hamiltonian $H$ which describes all the possible interactions within a molecular system of $N$ electrons and $M$ nuclei. Below is the full molecular Hamiltonian in atomic units, where $Z_i$ is the charge of nucleus $i$ and $M_i$ is its mass relative to the mass of an electron.
\begin{equation*}
    H =
    - \sum_{i=1}^{N} \frac{1}{2} \nabla^{2}_{i}
    - \sum_{i=1}^{M} \frac{1}{2M_i} \nabla^{2}_{i}
    - \sum_{i=1}^{N} \sum_{j=1}^{M} \frac{Z_j}{|r_{i} - R_{j}|}
    + \sum_{i=1}^{N} \sum_{j>i}^{N} \frac{1}{|r_{i} - r_{j}|}
    + \sum_{i=1}^{M} \sum_{j>i}^{M} \frac{Z_{i} Z_{j}}{|R_{i} - R_{j}|}
\end{equation*}

The first term corresponds to the kinetic energy of all electrons in the system and the second term corresponds to the total kinetic energy of all nuclei. The third term corresponds to the pairwise attractive Coulombic interactions between the $N$ electrons and $M$ nuclei, whilst the fourth and fifth terms correspond to all repulsive Coulombic interactions between electrons and nuclei respectively.

Using the Born-Oppenheimer approximation, we are able to simplify this problem to a purely electronic one. Motivated by the large difference in mass of electrons and nuclei, we can approximate the nuclei as stationary on the timescale of electronic motion such that the electronic wavefunction depends only parametrically on the nuclear coordinates. This allows us to express the full molecular wavefunction as an adiabatic separation as below.
\begin{equation*}
    \Phi_\text{total} =
    \psi_\text{elec}({\{r\}};\{R\}) \,
    \psi_\text{nuc}(\{R\})
\end{equation*}
Within this approximation, the nuclear kinetic energy term can be neglected and the nuclear repulsive term is considered to be constant. Since constants in eigenvalue equations have no effect on the eigenfunctions and simply add to the resulting eigenvalue, we will omit this too. The resulting equation is the electronic Hamiltonian for $N$ electrons. 
\begin{equation*}
    H =
    - \sum_{i=1}^{N} \frac{1}{2} \nabla^{2}_{i}
    - \sum_{i=1}^{N} \sum_{j=1}^{M} \frac{Z_j}{|r_{i} - R_{j}|}
    + \sum_{i=1}^{N} \sum_{j>i}^{N} \frac{1}{|r_{i} - r_{j}|}
\end{equation*}
We will only concern ourselves with the electronic Hamiltonian throughout the remainder of this text, simply referring to it as the Hamiltonian, $H$.

The solution to the eigenvalue equation involving the electronic Hamiltonian is the electronic wavefunction, which depends only parametrically on the nuclear coordinates. It is solved for fixed nuclear coordinates, such that different arrangements of nuclei yields different functions of the electronic coordinates. The total molecular energy can then be calculated by solving the electronic Schrödinger equation and including the constant repulsive nuclear term.
\begin{equation*}
    E_\text{total} = E_\text{elec} + \sum_{i=1}^{M} \sum_{j>i}^{M} \frac{Z_{i} Z_{j}}{|R_{i} - R_{j}|}
\end{equation*}

% something about how solving the molecular electronic Hamiltonian yields one-electron wavefunctions known as molecular orbitals. these depend only on the spatial coordinates.

\subsection{Many-Electron Wavefunctions}
something


\subsection{Second Quantisation}
In second quantisation, both observables and states (by acting on the vacuum state) are represented by operators, namely the creation and annhilation operators \cite{Helgaker2000}. In contrast to the standard formulation of quantum mechanics, operators in second quantisation incorporate the relevant Bose or Fermi statistics each time they act on a state, circumventing the need to keep track of symmetrised or antisymmetrised products of single-particle wavefunctions \cite{Fetter1972}. Put differently, the antisymmetry of an electronic wavefunction simply follows from the algebra of the creation and annhilation operators \cite{Helgaker2000}, which greatly simplifies the discussion of systems of many identical interacting fermions \cite{Fetter1972}.

\subsection{Occupation Number Representation}
The Fock space is a linear abstract vector space spanned by $N$ orthonormal occupation number vectors \cite{Helgaker2000}, each representing a single Slater determinant. Hence, given a basis of $N$ spin orbitals we can construct $2^N$ single Slater determinants, each corresponding to a single occupation number vector in the full Fock space.

TALK ABOUT HOW ON DIRAC NOTATION IS BINARY REPRESENTATION OF ON VECTOR
The occupation number vector for fermionic systems is succinctly denoted in Dirac notation as below, where the occupation number $f_j$ is 1 if spin orbital $j$ is occupied, and 0 if spin orbital $j$ is unnoccupied.
\begin{equation*}
    \ket\psi = \ket{f_{n-1} \,\, f_{n-2} \dots f_{1} \,\, f_{0}} \qquad \text{where } f_{j} \in {0, 1}
\end{equation*}
Whilst there is a one-to-one mapping between Slater determinants with canonically ordered spin orbitals and the occupation number vectors in the Fock space, it is important to distinguish between the two since, unlike the Slater determinants, the occupation number vectors have no spatial structure and are simply vectors in an abstract vector space. \cite{Helgaker2000}.

\begin{equation*}
    \ket{\psi_1} = \ket{0 \dots 1} =
    \begin{pmatrix} 1 \\ \vdots \\ 0 \end{pmatrix} \qquad
    \dots\qquad
    \ket{\psi_N} = \ket{1 \dots 1} =
    \begin{pmatrix} 0 \\ \vdots \\ 1 \end{pmatrix} \qquad
\end{equation*}

\subsection{Creation and Annhilation Operators}
Operators in second quantisation are constructed from the creation and annhilation operators $a_j^\dagger$ and $a_j$, where the subscripts $i$ and $j$ denote the spin orbital. $a_j^\dagger$ and $a_j$ are one another's Hermitian adjoints, and are not self-adjoint \cite{Helgaker2000}.

Taking the excitation of an electron from spin orbital 0 to spin orbital 1 as an example, we can construct the following excitation operator.
\begin{equation*}
    a_1^\dagger \, a_0 \ket{0 \dots 01} = \ket{0 \dots 10}
\end{equation*}
SHOW LADDER OPERATORS ACTING ON OPPOSITE STATES
Due to the fermionic exchange anti-symmetry imposed by the Pauli principle, the action of the creation and annhilation operators introduces a phase to the state that depends on the parity of the spin orbitals preceding the target spin orbital.
\begin{align*}
    a_j^\dagger \ket{f_{n-1} \dots
    f_{j+1},\,\, 0,\,\, f_{j-1} \dots f_0} &=
    (-1)^{\sum_{s=0}^{j-1} f_s}
    \ket{f_{n-1} \dots f_{j+1},\,\, 1,\,\, f_{j-1} \dots f_0} \\
    %
    a_j \ket{f_{n-1} \dots f_{j+1},\,\, 1,\,\, f_{j-1} \dots f_0} &=
    (-1)^{\sum_{s=0}^{j-1} f_s}
    \ket{f_{n-1} \dots f_{j+1},\,\, 0,\,\, f_{j-1} \dots f_0}
\end{align*}
In second quantisation, this exchange anti-symmetry requirement is accounted for by the anti-commutation relations of the creation and annhilation operators.
\begin{equation}
\begin{gathered}
    \{ \hat a_{j}, \hat a_{k} \} = 0 \qquad \qquad
    % = \hat a_{j} \hat a_{k} + \hat a_{k} \hat a_{j}
    %
    \{ \hat a_{j}^{\dagger}, \hat a_{k}^{\dagger} \} = 0 \qquad
    % = \hat a_{j}^\dagger \hat a_{k}^\dagger + \hat a_{k}^\dagger \hat a_{j}^\dagger \\
    %
    \{ \hat a_{j}, \hat a_{k}^{\dagger} \} = \delta_{jk} \hat{1}
    % = \hat a_{j} \hat a_{k}^\dagger + \hat a_{k}^\dagger \hat a_{j}
\end{gathered}
\end{equation}
That is, the phase factor required for the second quantised representation to be consistent with the first quantised representation is automatically kept track of by the anticommutation relations of the creation and annhilation operators \cite{Helgaker2000}.

\subsection{Second Quantised Hamiltonian}
The Hamiltonian in second quantisation is constructed from creation and annhilation operators as follows.
\begin{equation*}
    \hat H =
    \sum_{ij} h_{ij} a^\dagger_i a_j +
    \frac{1}{2} \sum_{ijkl} h_{ijkl} a^\dagger_i a^\dagger_j a_k a_l +
    h_\text{Nu}
\end{equation*}
Where the one-body matrix element $h_{ij}$ corresponds to the kinetic energy of an electron and its interaction energy with the nuclei.
\begin{equation*}
h_{ij} = \int^\infty_{-\infty} \psi^*_{i(x_1)} \left( - \frac{1}{2} \nabla^2 + \hat V_{(x_1)} \right) \psi_{j(x_1)} \,\, d^3 x_1 \\
\end{equation*}
The two-body matrix element $h_{ijkl}$ corresponds to the repulsive interaction between electrons $i$ and $j$.
\begin{equation*}
h_{ijkl} = \int^\infty_{-\infty} \int^\infty_{-\infty} \psi^*_{i(x_1)} \psi^*_{j(x_2)} \left( \frac{1}{|x_1 - x_2|} \right) \psi_{k(x_2)} \psi_{l(x_1)} \,\, d^3 x_1 d^3 x_2
\end{equation*}

$h_\text{Nu}$ is a constant corresponding to the repulsive interaction between nuclei. These matrix elements are computed classically, allowing us to compute only the inherently quantum aspects of the problem on a quantum computer.

