\chapter{\label{background}Background}
In this chapter, we will discuss the methods and concepts required to simulate fermionic systems on a quantum computer, as well as the notation that we will use throughout the text. Starting with ... we will introduce the ... and finally a variational quantum algorithm known as the Variational Quantum Eigensolver.

\section{Fermionic Simulation Scheme}
Let us first develop a general fermionic simulation scheme.

\begin{itemize}
    \item hamiltonian in first quantisation
    \item born oppenheimer approximation
    \item hamiltonian in second quantisation
    \item anti commutation relations of creation/annhilation operators
    \item introduction to unitary coupled cluster theory
\end{itemize}

In order to simulate a fermionic system on a quantum computer, we must map the fermionic state to a qubit state. This is usually done using the occupation number representation [REF SECTION]. We then act on the qubit state with unitary operations that represent the fermionic operations. In order not to violate the Pauli principle, we must choose a fermion-qubit mapping that preserves the fermionic anti-commutation relations. The most common mapping, and the one used throughout this text, is known as the Jordan-Wigner transformation [REF SECTION]. Then, by acting on the qubit state with the unitary qubit operator, we obtain the resultant qubit state, which in the occupation number representation, simply represents the fermionic wavefunction.

A successful simulation scheme is one that reproduces the action of the fermionic operator [CITE BRAVYI-KITAEV PAPER].
