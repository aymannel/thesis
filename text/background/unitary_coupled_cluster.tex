\section{Unitary Coupled Cluster}
\subsection{Second Quantisation}
In second quantisation, wavefunctions are represented in the occupation number representation.
\begin{equation*}
    \ket\psi = \ket{f_{n-1} \dots f_{0}} \qquad \text{where } f_{j} \in {0, 1}
\end{equation*}

Where $\ket\psi$ represents a Slater determinant usually constructed from Hartree-Fock spin orbitals.

This means that given $n$ spin-orbitals, there exist $2^n$ electronic basis states.
\begin{equation*}
    \ket\psi = \ket{0 \dots 10} =
    \begin{pmatrix} 0 \\ 1 \\ \vdots \\ \vdots \\ \vdots \\ 0 \end{pmatrix} \qquad
\end{equation*}

Fermionic interactions can be represented in terms of the creation and annhilation operators $a_j^\dagger$ and $a_j$. \bigskip

Consider the excitation of an electron in the lowest spin-orbital to the second spin-orbital.

\begin{equation*}
\begin{gathered}
    \ket{\psi_1} \rightarrow \ket{\psi_2} \\
    a_2^\dagger \, a_1 \ket{001} = \ket{010}
\end{gathered}
\end{equation*}

Importantly, due to the exchange anti-symmetry of fermions, the action of the creation/annhilation operators introduces a phase to the electronic wavefunction.

\begin{align*}
    a_j^\dagger \ket{f_{n-1} \dots
    f_{j+1},\,\, 0,\,\, f_{j-1} \dots f_0} &=
    (-1)^{\sum_{s=0}^{j-1} f_s}
    \ket{f_{n-1} \dots f_{j+1},\,\, 1,\,\, f_{j-1} \dots f_0} \\
    %
    a_j^\dagger \ket{f_{n-1} \dots f_{j+1},\,\, 1,\,\, f_{j-1}
    \dots f_0} &= \vec 0 \\
    %
    a_j \ket{f_{n-1} \dots f_{j+1},\,\, 1,\,\, f_{j-1} \dots f_0} &=
    (-1)^{\sum_{s=0}^{j-1} f_s}
    \ket{f_{n-1} \dots f_{j+1},\,\, 0,\,\, f_{j-1} \dots f_0} \\
    %
    a_j \ket{f_{n-1} \dots f_{j+1},\,\, 0,\,\, f_{j-1} \dots f_0} &= \vec 0
\end{align*}

The phase introduced depends on the \textbf{parity} of the spin-orbitals preceding spin-orbital $j$.

In the second-quantisation, this anti-symmetry is expressed in terms of the anti-commutation of creation and annhilation operators.
\begin{align*}
    \{ \hat a_{j}, \hat a_{k} \} &=
    \hat a_{j} \hat a_{k} + \hat a_{k} \hat a_{j} = 0 \\
    %
    \{ \hat a_{j}^{\dagger}, \hat a_{k}^{\dagger} \} &=
    \hat a_{j}^\dagger \hat a_{k}^\dagger + \hat a_{k}^\dagger \hat a_{j}^\dagger = 0 \\
    %
    \{ \hat a_{j}, \hat a_{k}^{\dagger} \} &= \hat a_{j} \hat a_{k}^\dagger + \hat a_{k}^\dagger \hat a_{j} = \delta_{jk} \hat{1}
\end{align*}

In fact, we find that the phase factor discussed in the previous section is automatically kept track of by these relations.

The Hamiltonian in second quantisation can also be expressed in terms of the creation and annhilation operators.

\begin{equation*}
    \hat H =
    \sum_{ij} h_{ij} a^\dagger_i a_j +
    \frac{1}{2} \sum_{ijkl} h_{ijkl} a^\dagger_i a^\dagger_j a_k a_l +
    h_\text{Nu}
\end{equation*}

Where $h_{ij}$ (one-electron overlap integral) and $h_{ijjk}$ (two-electron overlap integral) are computed classically.


\subsection{Coupled Cluster \& Unitary Coupled Cluster}

Within the traditional coupled-cluster framework, the ground electronic state is prepared by applying the CC operator to a reference state (usually Hartree-Fock).

\begin{equation*}
    \ket\psi = e^{\hat T} \ket{\phi_0}
\end{equation*}

Where $\hat T$ is the cluster excitation operator.

Quantum gates, however, must be unitary operators, so instead, we work within the UCC framework.

\begin{equation*}
    \ket\psi = e^{\hat T} \ket{\phi_0}
\end{equation*}

Where $\hat T$ is now an \textbf{anti-Hermitian} operator, and $e^{\hat T}$ is unitary.

In general, we can prepare exact electronic states by applying a sequence of $k$ parametrised unitary operators to our reference state.

\begin{equation*}
\begin{gathered}
    \ket\psi = \prod_i^k U_i(\theta_i) \ket{\phi_0} \\
    \text{Where $U_i(\theta_i)$ is a parametrised unitary operator}
\end{gathered}
\end{equation*}\smallskip

The parameters $\theta_i$ are then optimised to find the ground state energy.

General fermionic single and double excitation operators are defined as,
\begin{equation*}
    a_q^\dagger a_p \text{ and } a_r^\dagger a_s^\dagger a_q a_p
\end{equation*}

Exciting one electron from $p$ to $q$, and two electrons from $p, q$ to $r, s$ respectively.

Taking a linear combination of these, we obtain \textbf{anti-Hermitian} fermionic single and double excitation operators.
\begin{equation*}
\begin{gathered}
    \hat\kappa_p^q = a_q^\dagger a_p - a_p^\dagger a_q \\
    %
    \hat\kappa_{pq}^{rs} =
    a_r^\dagger a_s^\dagger a_q a_p - a_p^\dagger a_q^\dagger a_s a_r
\end{gathered}
\end{equation*}\smallskip

Such that upon exponentiating, we obtain \textbf{unitary} operators.

\begin{equation*}
    U^q_p = e^{\hat\kappa_p^q} \qquad
    %
    U_{pq}^{rs} = e^{\hat\kappa_{pq}^{rs}}
\end{equation*}

Recalling the Jordan-Wigner encoding for the creation and annhilation operators,

\begin{equation*}
    \hat a_j^+ = \frac{1}{2} (X - iY) \otimes Z^\rightarrow_{j-1} \qquad
    \hat a_j = \frac{1}{2} (X + iY) \otimes Z^\rightarrow_{j-1}
\end{equation*}

The anti-Hermitian fermionic single and double excitation operators $\kappa_p^q$ and $\kappa_{pq}^{rs}$
\begin{align*}
    F_p^q = \frac{i}{2} & (Y_p X_q - X_p Y_q) \prod_{k=p+1}^{q-1} Z_k \\
    %
    F_{pq}^{rs} = \frac{i}{8} (
      & X_p X_q Y_s X_r +
        Y_p X_q Y_s Y_r +
        X_p Y_q Y_s Y_r +
        X_p X_q X_s Y_r - \\
      & Y_p X_q X_s X_r -
        X_p Y_q X_s X_r -
        Y_p Y_q Y_s X_r -
        Y_p Y_q X_s Y_r )
    \prod_{k=p+1}^{q-1} Z_k
    \prod_{l=r+1}^{s-1} Z_l
\end{align*}

Multiplying by $\theta$ and exponentiating yields the parametrised unitary qubit operators,

\begin{equation*}
    U^q_p (\theta) =
    \text{exp} \left( i
    \frac{\theta}{2} (Y_p X_q - X_p Y_q) \prod_{k=p+1}^{q-1} Z_k \right)
\end{equation*}

\begin{equation*}
    U^{rs}_{pq} (\theta) = \text{exp} \left( i \frac{\theta}{8} (
    X_p X_q Y_s X_r
    + \dots -
    Y_p Y_q Y_s X_r -
    Y_p Y_q X_s Y_r )
    \prod_{k=p+1}^{q-1} Z_k
    \prod_{l=r+1}^{s-1} Z_l
    \right)
\end{equation*}

To summarise, we constructed anti-Hermitian single and double excitation operators from a linear combination of fermionic excitation operators,

\begin{equation*}
    \hat\kappa_p^q = a_q^\dagger a_p - a_p^\dagger a_q \qquad
    %
    \hat\kappa_{pq}^{rs} =
    a_r^\dagger a_s^\dagger a_q a_p - a_p^\dagger a_q^\dagger a_s a_r
\end{equation*}\smallskip

We then mapped these to qubit operators using the Jordan-Wigner transformation,
\begin{equation*}
    \hat\kappa_p^q \xrightarrow{\text{JW}} F_p^q \qquad\qquad
    \hat\kappa_{pq}^{rs} \xrightarrow{\text{JW}} F_{pq}^{rs}
\end{equation*}

And finally, we exponentiated to yield the parametrised unitary qubit operators.
\begin{equation*}
    U^q_p (\theta) = e^{\theta^q_p F_p^q} \qquad
    %
    U^{rs}_{pq}(\theta) = e^{\theta_{pq}^{rs} F_{pq}^{rs}}
\end{equation*}

