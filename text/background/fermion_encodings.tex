\section{Fermion-Qubit Encodings}
see \cite{Seeley2020}

The form of the occupation number representation vector and the qubit statevector suggests the following identification between electronic states and qubit states.
\begin{equation*}
    \ket{f_{n-1} \dots f_{0}} \quad\rightarrow\quad \ket{q_{n-1} \dots q_{0}}
\end{equation*}
That is, we allow each qubit to store the occupation number of a given spin-orbital. Hence, in order to actually simulate a Hamiltonian we must map the fermionic creation and annhilation operators onto qubit operators, and these operators must behave in the same way as their fermionic analogues.
\begin{equation*}
    \hat Q^+ \ket 0 = \ket 1 \qquad
    \hat Q^+ \ket 1 = 0 \qquad
    \hat Q \ket 1 = \ket 0 \qquad
    \hat Q \ket 0 = 0
\end{equation*}
The qubit operators must also preserve the fermionic anti-commutation relations in order to satisfy the Pauli antisymmetry requirement.
\begin{equation*}
    \{ \hat Q_{j}, \hat Q_{k} \} = 0 \qquad
    \{ \hat Q_{j}^{\dagger}, \hat Q_{k}^{\dagger} \} = 0 \qquad
    \{ \hat Q_{j}, \hat Q_{k}^{\dagger} \} = \delta_{jk}
\end{equation*}

One such qubit encoding is known as the Jordan-Wigner transformation. It expresses the fermionic creation and annhilation operators as a linear combination of the Pauli matrices.
\begin{equation*}
    \hat Q^+ = \ket 1 \bra 0 = \frac{1}{2} (X - iY) \qquad \hat Q = \ket 0 \bra 1 = \frac{1}{2} (X + iY) 
\end{equation*}

When dealing with \textbf{multiple-qubits}, we must also account for the occupation parity of the qubits preceding the target qubit $j$.
\begin{equation*}
    a_j^\dagger \ket{f_{n-1} \dots f_{j+1},\,\, 0,\,\, f_{j-1} \dots f_0} = (-1)^{\sum_{s=0}^{j-1} f_s} \ket{f_{n-1} \dots f_{j+1},\,\, 1,\,\, f_{j-1} \dots f_0}
\end{equation*}
We do this by introducing a string of Pauli Z operators that computes the parity of the qubits preceding the target qubit.
\begin{equation*}
\begin{gathered}
    \hat a_j^+ = \frac{1}{2} (X - iY) \prod_{k=1}^{j-1} Z_k \qquad
    \hat a_j = \frac{1}{2} (X + iY) \prod_{k=1}^{j-1} Z_k \\[2ex]
    \text{Where $\prod$ is the tensor product.}
\end{gathered}
\end{equation*}

A more compact notation is,
\begin{equation*}
    \hat a_j^+ = \frac{1}{2} (X - iY) \otimes Z^\rightarrow_{j-1} \qquad
    \hat a_j = \frac{1}{2} (X + iY) \otimes Z^\rightarrow_{j-1}
\end{equation*}
Where $Z^\rightarrow_{i}$ is the parity operator with eigenvalues $\pm 1$, and ensures the correct phase is added to the qubit state vector.
\begin{equation*}
    Z^\rightarrow_{i} = Z_i \otimes Z_{i-1} \otimes \dots \otimes Z_0
\end{equation*}
For instance, the creation operator $a^\dagger_3$ maps to the following Pauli string,
\begin{align*}
    \hat a_3^\dagger &=
    \frac{1}{2} (X_3 - iY_3) \otimes Z_2 \otimes Z_1 \otimes Z_0 \\
    %
    \hat a_3^\dagger &=
    \frac{1}{2} ( X_3 \otimes Z_2 \otimes Z_1 \otimes Z_0 ) -
    \frac{1}{2} i ( Y_3 \otimes Z_2 \otimes Z_1 \otimes Z_0 )
\end{align*}
Usually we drop the subscript specifying the orbital acted on.
