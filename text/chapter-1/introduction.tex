\chapter{Background}%
\label{background}

In this chapter, we will discuss the framework that we use to simulate fermionic systems on a quantum computer, as well as the notation that we will use throughout the text. Starting with Quantum Computation REF(quantum-computation) and Electronic Structure Theory REF(electronic-structure-theory), we will build up to unitary coupled cluster theory and the Variational Quantum Eigensolver REF(vqe).

Fermionic states can generally be represented on a quantum computer in the occupation number representation (section REF(second-quantisation)). That is, the state of each qubit is taken to represent the occupancy of each spin orbital. By representing the fermionic creation and annhilation operators in terms of qubit operators in a way that preserves the fermionic anticommutation relations, we can express the molecular Hamiltonian in terms of qubit operations.
