\chapter{Background}%
\label{background}

% In this chapter, we will develop the mathematical foundation used to simulate fermionic systems on quantum computers and explain the process that we followed during this year's work. Starting with Electronic Structure Theory (\ref{electronic-structure-theory}), we will build up to Unitary Coupled Cluster theory and the Variational Quantum Eigensolver REF, which serves as the framework that this research revolves around.

% Fermionic states can generally be represented on a quantum computer in the occupation number representation (section REF(second-quantisation)). That is, the state of each qubit is taken to represent the occupancy of each spin orbital. By representing the fermionic creation and annhilation operators in terms of qubit operators in a way that preserves the fermionic anticommutation relations, we can express the molecular Hamiltonian in terms of qubit operations.
