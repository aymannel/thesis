\chapter{\label{background}Background}
In this chapter, we will discuss the methods and concepts required to simulate fermionic systems on a quantum computer, as well as the notation that we will use throughout the text. Starting with electronic structure theory, we will introduce the the fundamentals of quantum computation, fermion-qubit encodings and finally Unitary Coupled Cluster theory.

Fermionic states can generally be represented on a quantum computer in the occupation number representation (section \ref{second-quantisation}). That is, the state of each qubit is taken to represent the occupancy of each spin orbital. Then, by representing the fermionic creation and annhilation operators in terms of qubit operators in a way that preserves the fermionic anticommutation relations (section \ref{fermion-qubit-encodings}), we are able to express the Hamiltonian in terms of qubit operations. To simulate this Hamiltonian, we obtain the corresponding unitary time evolution operator, usually via a number of Trotter steps.
