\section{Context \& Motivation}%
\label{context-motivation}

Using the ZX calculus, this thesis focuses on the diagrammatic representation of excitation operators used to account for the correlation in fermionic systems. The ZX calculus is diagrammatic language for reasoning about quantum processes \cite{Coecke2011} that has recently shown an increased usage in quantum computing and quantum simulation. The research represented in this thesis is conducted within the framework of the \textit{Variational Quantum Eigensolver (VQE)}, a promising hybrid quantum-classical algorithm for achieving quantum advantage on \textit{Noise Intermediate-Scale Quantum (NISQ)} devices \cite{Cerezo2020}.

The VQE algorithm divides the problem of estimating the ground-state energy of a molecule into two parts: computing the energy of some fermionic state on a quantum device, then classically optimising the quantum circuit representing the state until it converges to a good approximation of the true ground state. The \textit{Unitary-Product State (UPS)} ansatz, derived from the \textit{Unitary Coupled Cluster (UCC)} formulation of the wavefunction, allows us to implement parametrised representations of fermionic states on a quantum computer.

The VQE algorithm generates multiple distinct UPS ansätze, each yielding the same energy expectation value. Hence, while different sequences of unitary excitation operators are employed to rotate the reference state to a state approximating the true ground state, their identical energy expectation values suggests that they equivalently capture the correlation present in the ground state, implying that that it may be possible to demonstrate the equivalence between these UPS ansätze through algebraic manipulation.

In this context, the research presented in this thesis focuses on the diagrammatic representation of unitary excitation operators using the ZX calculus. Our initial goal was to identify a generalised structure for these excitation operators within the ZX calculus, anticipating that by doing so, we might discover a way of demonstrating the equivalence of different VQE ansätze with the same energy expectation value. By developing a representation for these excitation operators that is independent of specific architectural constraints, we sought to gain deeper insights into the structure of UPS ansätze and the nature of correlations in molecular quantum systems. This broader understanding could potentially lead to more efficient and effective quantum simulation algorithms. Furthermore, due to the poor fidelity of today's quantum computers, reducing the depth of quantum circuits is crucial to minimise errors in simulations. By identifying a generalised structure for the excitation operators used in the UPS ansatz, we aim to uncover novel methods for optimising ansätze that represent fermionic wavefunctions.

This led us to the work by Yordanov \textit{et al} \cite{Yordanov2020} and Kornell \textit{et al} \cite{Kornell2023}, which show that excitation operators can be rewritten in terms of controlled rotations, as well as the work by Cowtan \textit{et al.} \cite{Cowtan2020}, showing that commuting sets of Pauli gadgets can be diagonalised and optimally resynthesised. In this thesis, we demonstrate the correspondence between excitation operators and controlled rotations using the ZX calculus, potentially informing improved optimisation strategies.

While quantum chemistry is anticipated to be a principal application of quantum computing, it remains an area with limited engagement among Master's level researchers in Chemistry. Furthermore, the ZX calculus, despite its immense promise as a next-generation framework for studying quantum computing, has seen relatively low usage by quantum chemists. Therefore, we hope that this thesis, along with the tools developed herein, will help to lower the barrier to entry for future Master's and PhD students interested in quantum computing. By providing a solid theoretical foundation and practical insights, we aim to facilitate a smoother transition and foster greater interest in this rapidly developing field.

To summarise, we aim to identify a generalised representation of excitation operators within the ZX calculus to gain insights into the nature of correlations in molecular quantum systems. Then, using these insights, we seek to rationalise the expectation energy-equivalent outputs of VQE algorithms and discover more efficient implementations of UPS ansätze. Finally, we hope to lower the barrier to entry for quantum computing in the context of quantum chemistry.
