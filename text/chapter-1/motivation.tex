\subsection{Context \& Motivation}%
\label{context-motivation}

The Variational Quantum Eigensolver (VQE) is a promising hybrid quantum-classical algorithm for achieving quantum advantage on NISQ devices \cite{Cerezo2020}. Developed in 2014 by Peruzzo and McClean \textit{et al} \cite{Peruzzo2014}, the VQE algorithm divides the problem of estimating the ground-state energy of a molecule into two parts -- computing the energy of some fermionic state on a quantum device, then classically optimising the quantum circuit representing the state until it converges to a good approximation of the true ground state.

VQE algorithms implement fermionic states on quantum devices via the Unitary Coupled Cluster (UCC) ansatz \cite{Taube2006}. By preparing quantum states as a sequence of unitary excitation operations acting on some reference state, we are able to parametrically explore the Hilbert space of possible quantum states \cite{McClean2016}.

The Discretely and Continuously Optimised Variational Quantum Eigensolver (DISCO-VQE) is a specific type of VQE developed by Burton \textit{et al}. \cite{Burton2023}. This algorithm generates multiple distinct UCC ansätze, each yielding the same energy expectation value. This means that different sequences of unitary excitation operators are employed to rotate the reference state to a state approximating the true ground state. The identical energy expectation values of these states suggests that they equivalently capture the correlation present in the ground state, and that it may be possible to demonstrate the equivalence between these UCC ansätze through algebraic manipulation.

In this context, this thesis focuses on the diagrammatic representation of unitary excitation operators in the ZX calculus, a diagrammatic language for reasoning about quantum processes \cite{Coecke2011}. Our initial goal was to identify a generalised structure for these excitation operators within the ZX calculus, anticipating that by doing so, we might discover a way of demonstrating the equivalence of different VQE ansätze with the same energy expectation value. Through this, we aimed to uncover novel methods for optimising ansätze that represent fermionic wavefunctions.

Additionally, by developing a representation for these excitation operators that is independent of specific architectural constraints, we sought to gain deeper insights into the structure of UCC ansätze and the nature of correlations in molecular quantum systems. This broader understanding could potentially lead to more efficient and effective quantum simulations.

This led us to the work of Yordanov \textit{et al} \cite{Yordanov2020}, which shows that excitation operators can be re-expressed in terms of controlled rotations, and that of Cowtan \textit{et al.} \cite{Cowtan2020}, which shows that commuting sets of Pauli gadgets can be diagonalised. Throughout the course of our research, we were able to demonstrate diagrammatically the correspondence between these excitation operators and controlled rotations. Consequently, a significant portion of this thesis revolves around developing the diagrammatic techniques essential for replicating the findings of Yordanov \textit{et al} in the ZX calculus.

In addition to our research goals, this thesis aims to introduce the ZX calculus in the context of quantum chemistry. While quantum chemistry is anticipated to be a principal application of quantum computing, it remains an area with limited engagement among Master's level researchers in Chemistry. Therefore, we hope that this thesis, along with the tools developed herein, will help lower the barrier to entry for future Master's students interested in quantum computing. By providing a solid foundation and practical insights, we aim to facilitate a smoother transition and foster greater interest in this rapidly developing field.

\begin{enumerate}[itemsep=-5pt]
\item Use the ZX calculus to gain insights into the structure of UCC ansätze and the nature of correlations in molecular quantum systems.
\item Utilise these insights to rationalise the different outputs of VQE algorithms that yield the same energy.
\item Identify a general representation of excitation operators within the ZX calculus that is independent of specific architectural constraints.
\item Leverage this general representation to discover more efficient implementations of fermionic ansätze in terms of quantum resources.
\end{enumerate}

