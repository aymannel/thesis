\section{\label{electronic-structure-theory}Electronic Structure Theory}

\subsection{Electronic Structure Problem}

The main interest of electronic structure theory is finding approximate solutions to the eigenvalue equation of the full molecular Hamiltonian. Specifically, we seek solutions to the non-relativistic time-independent Schrödinger equation.

The full molecular Hamiltonian describes all of the interactions within a system of $N$ interacting electrons and $M$ nuclei. We are able to simplify the problem to an electronic one using the Born-Oppenheimer approximation. Motivated by the large difference in mass of electrons and nuclei, we can approximate nuclei as stationary on the timescale of electronic motion such that the electronic wavefunction depends only parametrically on the nuclear coordinates. Within this approximation, the nuclear kinetic energy term can be neglected and the nuclear repulsive term is considered to be constant. Following this, we obtain the electronic Hamiltonian for $N$ electrons.

\begin{figure}[H]
\begin{equation*}
    H =
    - \sum_{i=1}^{N} \frac{1}{2} \nabla^{2}_{i}
    - \sum_{i=1}^{N} \sum_{j=1}^{M} \frac{Z_j}{|r_{i} - R_{j}|}
    + \sum_{i=1}^{N} \sum_{j>i}^{N} \frac{1}{|r_{i} - r_{j}|}
\end{equation*}
\caption{Electronic molecular Hamiltonian in atomic units.}
\end{figure}

The first term corresponds to the kinetic energy of the $N$ electrons in the system, the second term corresponds to the pairwise attractive Coulombic interactions between the $N$ electrons and $M$ nuclei and the final term corresponds to the repulsive Coulombic interactions between $N$ electrons in the system. Throughout the remainder of this text, we concern ourselves only with the electronic Hamiltonian, simply referring to it as the Hamiltonian, $H$. The solution to the eigenvalue equation involving the electronic Hamiltonian is the electronic wavefunction, which depends only parametrically on the nuclear coordinates. It is solved for fixed nuclear coordinates, such that different arrangements of nuclei yields different functions of the electronic coordinates. The total molecular energy can be calculated by solving the electronic Schrödinger equation and including a constant nuclear term.

\subsection{Many-Electron Wavefunctions}%
\label{many-electron-wavefunctions}

The many-electron wavefunction, which describes all the electrons in given molecular system, must satisfy the Pauli principle. This is an independent postulate of quantum mechanics that requires the many-electron wavefunction to be antisymmetric with respect to the exchange of any two fermions.

A spatial molecular orbital is defined as a one-particle function of the position vector, spanning the whole molecule. The spatial orbitals form an orthonormal set $\{\psi_i(\mathbf{r})\}$, which if complete can be used to expand any arbitrary single-particle molecular wavefunction, that is, an arbitrary single-particle function of the position vector. In practice, only a finite set of such orbitals is available to us, spanning only a subspace of the complete space. Hence, wavefunctions expanded using this finite set are described as being `exact' only within the subspace that they span.

We now introduce the spin orbitals $\{\phi_i(\mathbf{x})\}$, that is, the set of functions of the composite coordinate $\mathbf{x}$, which describes both the spin and spatial distribution of an electron. Given a set of $K$ spatial orbitals, we can construct $2K$ spin orbitals by taking their product with the orthonormal spin functions $\alpha(\omega)$ and $\beta(\omega)$. Whilst the Hamiltonian operator makes no reference to spin, it is a necessary component when constructing many-electron wavefunctions in order to correctly antisymmetrise the wavefunction with respect to fermion exchange. Constructing the antisymmetric many-electron wavefunction from a finite set of spin orbitals amounts to taking the appropriate linear combinations of symmetric products of $N$ spin orbitals. A general procedure for this is achieved by constructing a Slater determinant from the finite set of spin orbitals, where each row relates to the electron coordinate $\mathbf{x_n}$ and each column corresponds to a particular spin orbital $\phi_i$ \cite{Atilla1996}. 

\begin{figure}[H]
\centering
\begin{equation*}
\psi(\mathbf{x}_1, \mathbf{x}_2) =
%
\frac{1}{\sqrt{N!}} \begin{vmatrix}
\phi_i (\mathbf{x}_1) & \phi_j (\mathbf{x}_1) & \dots & \phi_k (\mathbf{x}_1) \\
\phi_i (\mathbf{x}_2) & \phi_j (\mathbf{x}_2) & \dots & \phi_k (\mathbf{x}_2) \\
\vdots & \vdots &   & \vdots \\
\phi_i (\mathbf{x}_N) & \phi_j (\mathbf{x}_N) & \dots & \phi_k (\mathbf{x}_N)
\end{vmatrix}
\end{equation*}
\caption{Slater determinant representing an antisymmetrised $N$-electron wavefunction.}
\end{figure}

The Hartree-Fock method yields a set of orthonormal spin orbitals, which when used to construct a single Slater determinant, gives the best variational approximation to the ground state of a system \cite{Atilla1996}. By treating electron-electron repulsion in an average way, the Hartree-Fock approximation allows us to iteratively solve the Hartree-Fock equation for spin orbitals until they become the same as the eigenfunctions of the Fock operator. This is known as the \textit{Self-Consistent Field (SCF)} method and is an elegant starting point for finding approximate solutions to the many-electron wavefunction.

\subsection{\label{second-quantisation}Second Quantisation}

In second quantisation, both observables and states are represented by creation and annihilation operators. Unlike the standard formulation of quantum mechanics, these operators inherently incorporate Bose or Fermi statistics, eliminating the need to track symmetrised or antisymmetrised products of single-particle wavefunctions. Put differently, the antisymmetry of an electronic wavefunction follows from the algebra of these operators, which greatly simplifies the discussion of many identical interacting fermions \cite{Helgaker2000}, \cite{Fetter1972}.

The Fock space is a linear abstract vector space spanned by $N$ orthonormal occupation number vectors, each representing a single Slater determinant \cite{Helgaker2000}. Hence, given a basis of $N$ spin orbitals we can construct $2^N$ single Slater determinants, each corresponding to a single occupation number vector in the full Fock space. The occupation number representation of fermionic states is succinctly denoted in Dirac notation as below, where the occupation number $f_j$ is 1 if spin orbital $j$ is occupied, and 0 if spin orbital $j$ is unoccupied.
\begin{equation*}
    \ket\psi = \ket{f_{n-1} \,\, f_{n-2} \dots f_{1} \,\, f_{0}} \qquad \text{where } f_{j} \in {0, 1}
\end{equation*}
Whilst there is a one-to-one mapping between Slater determinants with canonically ordered spin orbitals and the occupation number vectors in the Fock space, it is important to distinguish between the two since, unlike the Slater determinants, the occupation number vectors have no spatial structure and are simply vectors in an abstract vector space \cite{Helgaker2000}.

Operators in second quantisation are constructed from the creation and annihilation operators $a_j^\dagger$ and $a_j$, where the subscripts $i$ and $j$ denote the spin orbital. $a_j^\dagger$ and $a_j$ are one another's Hermitian adjoints, and are not self-adjoint \cite{Helgaker2000}. Due to the fermionic exchange anti-symmetry imposed by the Pauli principle, the action of the creation and annihilation operators introduces a phase to the state that depends on the parity of the spin orbitals preceding the target spin orbital. This phase factor is automatically kept track of by the creation and annihilation operators' anti-commutation relations \cite{Helgaker2000}.

\begin{figure}[H]
    \centering
    \begin{equation*}
        \{ \hat a_{j}, \hat a_{k} \} = 0 \qquad \qquad
        \{ \hat a_{j}^{\dagger}, \hat a_{k}^{\dagger} \} = 0 \qquad
        \{ \hat a_{j}, \hat a_{k}^{\dagger} \} = \delta_{jk} \hat{1}
    \end{equation*}
    \caption{Anti-commutation relations of the creation and annihilation operators.}
    \label{anticommutation-relations}
\end{figure}

The Hamiltonian in second quantisation can be expressed as follows, where the one-body matrix element $h_{ij}$ corresponds to the kinetic energy of an electron and its interaction energy with the nuclei, and the two-body matrix element $h_{ijkl}$ corresponds to the repulsive interaction between electrons $i$ and $j$.

\begin{figure}[H]
    \centering
    \begin{equation*}
        \hat H =
        \sum_{ij} h_{ij} a^\dagger_i a_j +
        \frac{1}{2} \sum_{ijkl} h_{ijkl} a^\dagger_i a^\dagger_j a_k a_l +
        h_\text{Nu}
    \end{equation*}
    \caption{Hamiltonian in second quantisation.}
    \label{hamiltonian}
\end{figure}

These matrix elements are then computed classically, allowing us to simulate only the inherently quantum aspects of the problem on a quantum computer.
\begin{gather*}
h_{ij} = \int^\infty_{-\infty} \psi^*_{i(x_1)} \left( - \frac{1}{2} \nabla^2 + \hat V_{(x_1)} \right) \psi_{j(x_1)} \,\, d^3 x_1 \\[4ex]
%
h_{ijkl} = \int^\infty_{-\infty} \int^\infty_{-\infty} \psi^*_{i(x_1)} \psi^*_{j(x_2)} \left( \frac{1}{|x_1 - x_2|} \right) \psi_{k(x_2)} \psi_{l(x_1)} \,\, d^3 x_1 d^3 x_2
\end{gather*}
