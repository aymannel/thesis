\section{Quantum Computation}

\subsection{Introduction to Qubits}

In contrast to classical computation, where bits form the basis for encoding information, quantum computation makes use of quantum bits known as qubits. There are many physical implementations of qubits, however, in the theory of quantum computation, it suffices to think of them as purely mathematical objects.

The principal difference between the two, is that qubits can take values that are linear combinations of the computational basis. In quantum computing, the $\ket 0$ and $\ket 1$ states form the computational basis. They are orthonormal vectors in a two-dimensional complex Hilbert space $\mathbb{C}^2$. We can depict these computational basis states on a Bloch sphere. Note that the Bloch space does not represent the complex Hilbert space itself, but rather the Bloch space.
\begin{figure}[H]
\centering
    \begin{minipage}{.4\textwidth}
      \centering
      \includegraphics[width=0.5\linewidth]{chapter-1/zero}
      \caption{$\ket 0$ basis state}
    \end{minipage}%
    \begin{minipage}{.4\textwidth}
      \centering
      \includegraphics[width=0.5\linewidth]{chapter-1/one}
      \caption{$\ket 1$ basis state}
    \end{minipage}
\end{figure}

More generally, we can choose any pair of orthonormal states to form our computational basis. On the bloch sphere, this corresponds to any two vectors pointing in opposite directions. One such computational basis is the $\ket +$/$\ket -$ basis.
\begin{equation*}
\begin{gathered}
    \ket + = \frac{1}{\sqrt{2}} (\ket 0 + \ket 1) \qquad\qquad
    \ket - = \frac{1}{\sqrt{2}} (\ket 0 - \ket 1)
\end{gathered}
\end{equation*}

\begin{figure}[H]
\centering
    \begin{minipage}{.4\textwidth}
      \centering
      \includegraphics[width=0.5\linewidth]{chapter-1/plus}
      \caption{$\ket +$ state basis state}
    \end{minipage}%
    \begin{minipage}{.4\textwidth}
      \centering
      \includegraphics[width=0.5\linewidth]{chapter-1/minus}
      \caption{$\ket -$ state basis state}
    \end{minipage}
\end{figure}

More generally, any qubit $\ket\psi$ can be represented as complex linear combination of the chosen basis, provided that the qubit state vector is normalised.
\begin{equation*}
\begin{gathered}
    \ket \psi = \alpha\ket 0 + \beta\ket 1 \qquad
    |\alpha|^2 + |\beta|^2 = 1 \qquad
    \alpha, \beta \in \mathbb{C}
\end{gathered}
\end{equation*}

\subsection{Multiple Qubit States}
Suppose we have $n$ qubits. By taking the Kronecker product, we can construct $2^n$ computational basis states.
\begin{figure}[H]
\centering
\begin{gather*}
\ket{00 \dots 00} =
\ket{0}_n \otimes \ket{0}_{n-1} \otimes \dots
\otimes \ket{0}_1 \otimes \ket{0}_0 \\[1ex]
%
\dots \\
%
\ket{11 \dots 11} =
\ket{1}_n \otimes \ket{1}_{n-1} \otimes \dots
\otimes \ket{1}_1 \otimes \ket{1}_0
\end{gather*}
\caption{$2^n$ computational basis states.}
\end{figure}

It follows then that any complex linear combination of the computational basis states is also a valid qubit state.
\begin{equation*}
\ket\psi =
\alpha_{00 \dots 00}\ket{00 \dots 00} +
\alpha_{00 \dots 01}\ket{00 \dots 01} +
\dots +
\alpha_{11 \dots 11}\ket{11 \dots 11}
\end{equation*}

Whilst the Bloch sphere representation of a single qubit is incredibly useful, there is no easy generalisation of the Bloch sphere for multiple qubit states \cite{Nielsen2012}.
