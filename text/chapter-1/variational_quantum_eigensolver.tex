\section{The Variational Quantum Eigensolver}%
\label{vqe}

% Developed in 2014 by Peruzzo and McClean \textit{et al} \cite{Peruzzo2014}, 

%  VQE algorithms implement fermionic states on quantum devices via the Unitary Coupled Cluster (UCC) ansatz \cite{Taube2006}. By preparing quantum states as a sequence of unitary excitation operations acting on some reference state, we are able to parametrically explore the Hilbert space of possible quantum states \cite{McClean2016}.

% Consequently, a significant portion of this thesis revolves around developing the diagrammatic techniques essential for replicating the findings of Yordanov \textit{et al} in the ZX calculus.

% specifically the \textit{Discretely and Continuously Optimised Variational Quantum Eigensolver (DISCO-VQE)},


In this chapter, we introduce the \textit{Variational Quantum Eigensolver (VQE)} algorithm, a particular class of variational quantum algorithms used to estimate the ground state energy of molecular systems. The research presented in this thesis is conducted within this framework. As providing an overview of the VQE algorithm, we introduce the fundamentals of quantum computing and the \textit{Unitary Product State (UPS) ansatz} used to represent fermionic wavefunctions on quantum devices.

The VQE algorithm consists of a quantum subroutine, a \textit{Parametrised Quantum Circuit (PQC)}, that implements some UPS ansatz, and a classical subroutine that classically optimises the ansatz until it converges to the best approximation of the true ground state. PQCs are similar to classical neural networks in concept, but by definition, correspond to $2^n \times 2^n$ unitary maps, where $n$ is the number of qubits, and hence have the same number of inputs as outputs \cite{Yeung2020}.

The input state for the PQC is the reference state that the unitary operator $U(\bm\theta)$ acts on, which in the case of the single Slater determinant Hartree-Fock state, is encoded as a pure quantum state $\ket{\psi_0}$. The output of the PQC is then an entangled state, that is, some linear combination of vectors in the Fock space, that captures the correlation present in true ground state of the molecular system of interest.

Upon measuring the PQC output state, it collapses into a single vector in the Fock space, with a probability defined by that vector's weight in the UPS ansatz. The quantum subroutine computes the energy expectation value of the UPS ansatz via a quantum circuit consisting of the PQC and the Hamiltonian for the system.

\includezxdiagram{chapter-1/expectation}{0.7}
\begin{equation*}
    E(\bm\theta) = \bra{0} U^\dagger(\bm\theta) H U(\bm\theta) \ket{0} 
\end{equation*}

For the purposes of this thesis, we are interested in a variant of the VQE algorithm developed by Burton \textit{et al} \cite{Burton2023} known as the \textit{Discretely and Continuously Optimised Variational Quantum Eigensolver (DISCO-VQE)}. We implement a fermionic state on a quantum device as a sequence of quantum gates representing unitary excitation operators that act on some reference state. The DISCO-VQE algorithm then finds approximate solutions to the fermionic ground state by both (\textit{discretely}) optimising the sequence of fermionic excitation operators, chosen from a finite operator pool, and (\textit{continuously}) optimising their parameters. This allows us to parametrically explore the Hilbert space of possible quantum states until we find a good approximation of the true fermionic ground state \cite{Taube2006}. The DISCO-VQE algorithm, therefore, discovers accurate representations of fermionic wavefunctions using a minimal number of parameters \cite{Burton2023}.

%%%

\subsection{Unitary Product State Ansatz}%
\label{ups-ansatz}

As suggested by Peruzzo \textit{et al} \cite{Peruzzo2014}, the UCC formulation of a wavefunction can be efficiently implemented on a quantum device in terms of quantum gates. We refer to the UCC ansatz $\ket{\psi_{\text{UCC}}(\bm\theta)}$ as some unitary excitation operator $U_\text{UCC}(\bm\theta)$ acting on a reference state, usually a single-Slater determinant Hartree-Fock state $\ket{\psi_0}$ obtained via the self-consistent field method.
\begin{equation*}
    \ket{\psi_{\text{UCC}}(\bm\theta)} = U_\text{UCC}(\bm\theta) \ket{\psi_0} =
    e^{\hat T(\bm\theta) - \hat T^\dagger(\bm\theta)} \ket{\psi_0}
\end{equation*}

The operator $\hat T(\bm\theta)$ is a linear combination of fermionic second-quantised excitation operators, parametrised by coupled cluster amplitudes $\bm\theta$. The exponential of the anti-Hermitian operator $\hat T(\bm\theta) - \hat T^\dagger(\bm\theta)$ is therefore, by definition, unitary. 
\begin{equation*}
\hat T(\bm{\theta}) - \hat T^{\dagger}(\bm{\theta}) =
\sum_{i, a} \theta^a_i (a^\dagger_i a_a - a^\dagger_a a_i) + 
\sum_{i, j, a, b} \theta^{ab}_{ij} (a^\dagger_i a^\dagger_j a_a a_b - a^\dagger_a a^\dagger_b a_i a_j) + \dots
\end{equation*}

Where $i, j$ indexes occupied spin orbitals and $a, b$ indexes virtual, or unoccupied, spin orbitals. The \textit{Singles-Doubles} formulation of the UCC ansatz has been shown by Evangelista \textit{et al} to exactly parametrise any state, \cite{Evangelista2019}. However, since the terms of the unitary operator $U_\text{UCC}(\bm\theta)$ do not commute, it cannot be directly implemented on a quantum computer. Instead, by invoking the Trotter formula to approximate the unitary with a single Trotter step, since our focus is on the NISQ setting \cite{Cowtan2020}, we define the \textit{Unitary Product State (UPS)} ansatz $\ket{\psi(\bm\theta)}$ as the following product of $k$ parametrised unitary excitation operators \cite{Burton2023}.

\begin{figure}[H]
    \centering
    \begin{equation*}
        \ket{\psi(\bm\theta)} = \prod_{m=1}^k U_m(\theta_m) \ket{\psi_0} \qquad
        U_m(\theta_m) = e^{\theta_m (\tau_m - \tau_m^\dagger)}
    \end{equation*}
    \caption{Parametrised $k$-UPS ansatz.}
\end{figure}

Where $m$ indexes all possible excitations and $\tau_m - \tau_m^\dagger$ corresponds to anti-Hermitian fermionic excitation operators in second quantisation. For the remainder of this thesis, we refer to fermionic excitation operators as the the exponentials of these anti-Hermitian operators, $U_m(\theta_m)$. As these fermionic excitation operators are derived from second-quantised operators, they preserve the fermionic anti-symmetry and particle number of the reference state \cite{Burton2023}.

The operators are then mapped to quantum gates using the Jordan-Wigner transformation (see Chapter \ref{excitation-operators}) before being implemented on a quantum device. Within the DISCO-VQE framework, we consider only generalised \textit{single} (one-body) and \textit{double} (two-body) fermionic excitation (operators), which, as shown by Burton \textit{et al} \cite{Burton2023}, achieves universality provided that we combine enough suitably-ordered one-body and two-body excitation operators. In other words, the UPS ansatz is sufficient to represent any vector in the Hilbert space.
