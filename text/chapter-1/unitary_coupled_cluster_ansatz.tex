\subsection{Unitary Coupled Cluster Ansatz}%
\label{unitary-coupled-cluster-ansatz}

As suggested by Peruzzo \textit{et al} \cite{Peruzzo2014}, the Unitary Coupled Cluster (UCC) formulation of a wavefunction can be efficiently implemented on a quantum computer using quantum gates. We will refer to this implementation as the UCC ansatz $\ket{\psi(\bm\theta)}$, defining it as some unitary excitation operator $U(\bm\theta)$ acting on a reference state.
\begin{equation*}
    \ket{\psi(\bm\theta)} = U(\bm\theta) \ket{\psi_0} =
    e^{\hat T(\bm\theta) - \hat T^\dagger(\bm\theta)} \ket{\psi_0}
\end{equation*}

Where $\ket{\psi_0}$ is a single reference Slater determinant, usually the Hartree-Fock groundstate obtained via the self-consistent field method. The operator $\hat T(\bm\theta)$ is a linear combination of fermionic excitation operators, parametrised by coupled cluster amplitudes $\bm\theta$. The exponential of the anti-Hermitian linear combination, $\hat T(\bm\theta) - \hat T^\dagger(\bm\theta)$, is therefore, by definition, unitary. 
\begin{equation*}
\hat T(\bm{\theta}) - \hat T^{\dagger}(\bm{\theta}) =
\sum_{i, a} \theta^a_i (a^\dagger_i a_a - a^\dagger_a a_i) + 
\sum_{i, j, a, b} \theta^{ab}_{ij} (a^\dagger_i a^\dagger_j a_a a_b - a^\dagger_a a^\dagger_b a_i a_j) + \dots
\end{equation*}

Where $i, j$ indexes occupied spin orbitals and $a, b$ indexes virtual, or unoccupied, spin orbitals. The resulting unitary operator $U(\bm\theta)$ cannot be directly implemented on a quantum computer since the terms of the excitation operator do not commute. Instead, we must invoke the Trotter formula to approximate the unitary. Taking a single Trotter step $\rho=1$, since our focus is on the NISQ setting \cite{Cowtan2020}, we define the UCC ansatz as the following product of $k$ parametrised unitary operators.
\begin{equation*}
    \ket{\psi(\bm\theta)} = \prod_{m=1}^k U_m(\theta_m) \ket{\psi_0} \qquad
    U_m(\theta_m) = e^{\theta_m (\tau_m - \tau_m^\dagger)}
\end{equation*}
Where $m$ indexes all possible excitations and $\tau_m - \tau_m^\dagger$ corresponds to our anti-Hermitian fermionic excitation operators. It has been shown by Evangelista \textit{et al} \cite{Evangelista2019} that this UCC operator can exactly parametrise any state. In practice, we truncate the possible fermionic excitations to include only single and double excitations, yielding the popular UCCSD ansatz \cite{Chan2021}.

