\section{Creating Gadgets and Circuits}

We begin by introducing the \lstinline{Gadget} class, which we use to represent Pauli gadgets. The \lstinline{Gadget} class takes a Pauli string and a phase as inputs. For instance, we instantiate the $\text{exp} \left[ - i\frac{\pi}{4} \left(Y \otimes Z \otimes X \right) \right]$ gadget as follows.

\includejupyter{chapter-6/gadget}{chapter-6/gadget_zx}{0.3}

By setting \lstinline{as_gadget=False}, we expand the gadget in quantum circuit notation.

\includejupyter{chapter-6/expanded}{chapter-6/expanded_zx}{0.55}

We can construct a circuit of Pauli gadgets using the \lstinline{GadgetCircuit} class, providing it with an ordered list of \lstinline{Gadget()} objects. For instance, we implement a one-body excitation operator below, as discussed in Chapter \ref{excitation-operators}.

\includejupyter{chapter-6/circuit}{chapter-6/circuit_zx}{0.45}

We now introduce the classes implementing the standard quantum gates. As we will see in Section \ref{manipulating-circuits}, ZxFermion encodes the logic describing the interaction of the these gates with Pauli gadgets, allowing us to replicate the results in Chapter \ref{excitation-operators}. The CNOT and CZ gates are implemented by the \lstinline{CX} and \lstinline{CZ} classes respectively. We can specify the control and target qubits using the \lstinline{control} and \lstinline{target} parameters. When not specified, these parameters default to \lstinline{control=0} and \lstinline{target=1}.

\includejupyter{chapter-6/cx_cz}{chapter-6/cx_cz_zx}{0.3}

The Pauli $Z$ and Pauli $X$ gates are implemented by the \lstinline{Z} and \lstinline{X} classes respectively. We specify the target qubit using the \lstinline{qubit} parameter (defaults to \lstinline{qubit=0}). 

\includejupyter{chapter-6/paulis}{chapter-6/paulis_zx}{0.25}

Similarly, the single-qubit Clifford gates are implemented by the \lstinline{ZPlus}, \lstinline{ZMinus}, \lstinline{XPlus} and \lstinline{XMinus} classes. Below we implement the $Y \otimes Y$ Pauli string (Section \ref{clifford-conjugation}).

\includejupyter{chapter-6/cliffords}{chapter-6/cliffords_zx}{0.25}

Finally, we have the Hadamard gate, which is implemented by the \lstinline{H} class.

\includejupyter{chapter-6/hadamard}{chapter-6/hadamard_zx}{0.15}
