\section{Creating Gadgets and Circuits}

In this section, we will introduce the \lstinline{Gadget} class, which we use to represent Pauli gadgets. We provide a Pauli string and a phase to instantiate a \lstinline{Gadget()} object.

\includejupyter{chapter-5/gadget}{chapter-5/gadget_zx}{0.28}

By setting the \lstinline{as_gadget} option to \lstinline{False}, we can view the gadget in its expanded form, that is, in quantum circuit notation.

\includejupyter{chapter-5/expanded}{chapter-5/expanded_zx}{0.5}

We can construct a circuit of Pauli gadgets using the \lstinline{GadgetCircuit} class. The underlying data structure for this class is simply an ordered list.

\includejupyter{chapter-5/circuit}{chapter-5/circuit_zx}{0.42}

ZxFermion also implements standard quantum gates via the \lstinline{CX}, \lstinline{CZ}, \lstinline{X}, \lstinline{Z}, \lstinline{XPlus}, \lstinline{XMinus}, \lstinline{ZPlus} and \lstinline{ZMinus} classes. As we will see in the next section, we have implemented the logic describing the interaction of Pauli gadgets with these gates.

% The \lstinline{GadgetCircuit} class can also take standard quantum gates, where the \lstinline{stack} parameter stacks the individual gates in the circuit for a better representation.

% \includejupyter{chapter-5/stacked}{chapter-5/stacked_zx}{0.3}

