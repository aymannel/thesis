\section{Manipulating Circuits}%
\label{manipulating-circuits}

The \lstinline{GadgetCircuit} class offers the ability to manipulate circuits containing Pauli gadgets via the \lstinline{apply()} method. This method takes a quantum gate object as its input and inserts it along with its adjoint, into the circuit. The \lstinline{start} and \lstinline{end} parameters allow us to specify the insertion positions. For instance, below, we prepend two adjacent CNOT gates to the $\text{exp} \left[ - i\frac{1}{4} \left(Y \otimes Z \right) \right]$ gadget.

\includejupyter{chapter-6/apply1}{chapter-6/apply1_zx}{0.4}

If no insertion positions are specified, the specified quantum gate, and its adjoint, are added at the start and end of the circuit respectively. The \lstinline{apply()} method then applies the relevant commutation relations (see Section \ref{clifford-commutation-relations}) as required.

\includejupyter{chapter-6/apply2}{chapter-6/apply2_zx}{0.4}

We could have chosen to insert any of the previously mentioned quantum gates. The commutation logic is then implemented using Stim's \lstinline{Tableau} class, ensuring that the correct transformation is applied. Below we conjugate with the CZ gate.

\includejupyter{chapter-6/apply_cz}{chapter-6/apply_cz_zx}{0.4}

The \lstinline{GadgetCircuit} class offers the ability to manipulate circuits containing multiple gadgets simultaneously. Below, we instantiate the minimal one-body excitation operator discussed in Figure \ref{minimal-one-body}.

\includejupyter{chapter-6/single1}{chapter-6/single1_zx}{0.45}

It is then a simple matter of conjugating with CNOT gates to faithfully replicate the derivation revealing a singly-controlled $Y$ rotation as shown in Section \ref{operator-controlled-rotations}.

\includejupyter{chapter-6/single2}{chapter-6/single2_zx}{0.58}

