\section{Rewrite Rules}%
\label{rewrite-rules}

\notation{We refer to the rules by some shorthand notation above equal signs. Note that this could refer to applying the rule in either direction.}

\subsection{Spider Fusion}%
The most fundamental rule of the ZX calculus is the \textit{spider fusion rule} (\textit{f}). It states that spiders of the same colour connected by one or more wires fuse and their phases add modulo $2\pi$ \cite{Wetering2020}.

\begin{figure}[H]
    \centering
    \includegraphics[width=0.75\textwidth]{chapter-2/fusion}
    \caption{Spider fusion rule for $Z$ spiders (left) and $X$ spiders (right).}
    \label{spider-fusion}
\end{figure}

It is the generalisation of adding the phases of successive rotations of the Bloch sphere. We can use this rule to show that $Z$ rotations commute through CNOT controls, and that $X$ rotations commute through CNOT targets.

% \vspace{5pt}
\includezxdiagram{chapter-2/cnot_commutation}{0.8}

%%%

\subsection{Identity Removal}%

The \textit{identity rule} (\textit{id}) states that any two-legged spider with no phase ($\alpha = 0$) is equivalent to a rotation by 0 radians, or identity.

\begin{figure}[H]
    \centering
    \includegraphics[width=0.6\textwidth]{chapter-2/identity}
    \caption{Identity removal rule.}
    \label{identity}
\end{figure}

Combining this with the spider fusion rule (\ref{spider-fusion}), we see that two successive rotations with opposite phases is equivalent to an empty wire.

\includezxdiagram{chapter-2/cancelling_rotations}{0.7}

%%%

\subsection{State Copy and $\pi$ Copy Rules}%

We can depict the $Z$ and $X$ eigenstates (\ref{z-eigenstates-zx}, \ref{x-eigenstates-zx}) by assigning a phase of $\pi$ multiplied by a Boolean variable $a$ (0 or 1) to an $X$ or $Z$ spider, respectively \cite{Wetering2020}.

\includezxdiagramtext{chapter-2/boolean_x}{0.075}{
\ket 0 \text{ where $a = 0$ and }
\ket 1 \text{ where $a = 1$}}
%
\includezxdiagramtext{chapter-2/boolean_z}{0.075}{
\ket + \text{ where $a = 0$ and }
\ket - \text{ where $a = 1$}}

The $\pi$ \textit{copy rule} (\textit{c}) states that when a Pauli $Z$ or Pauli $X$ gate is pushed through a spider of the opposite colour, it is copied on all other legs and negates the spider's phase. A similar \textit{state copy rule} (\textit{c}) applies to the $Z$ and $X$ eigenstates.

\begin{figure}[H]
    \centering
    \includegraphics[width=0.9\textwidth]{chapter-2/pi_state_copy}
    \caption{$\pi$ copy (left) and state copy (right) rules for Pauli $Z$ gate and $Z$ eigenstates.}
    \label{state-copy}
    \label{pi-copy}
\end{figure}

%%%

\subsection{Hadamard Rules}

Using that the Hadamard gate is both unitary and Hermitian, we define the \textit{Hadamard self-inverse rule} (\textit{hi}) as follows.

\begin{figure}[H]
    \centering
    \includezxdiagram{chapter-2/hadamard_inverse}{0.42}
    \caption{Hadamard self-inverse rule.}
    \label{hadamard-self-inverse}
\end{figure}

Recalling that the Hadamard generator changes the colour of a spider and is self-inverse, we define the \textit{Hadamard commutation rule} (\textit{hc}).

\begin{figure}[H]
    \centering
    \includezxdiagram{chapter-2/hadamard_copy}{0.7}
    \caption{Hadamard commutation rule.}
    \label{hadamard-commutation}
\end{figure}

%%%

\subsection{Bialgebra Rule}

Using the spider fusion rule (\ref{spider-fusion}), we can show that a spider with two inputs and one output behaves like a classical XOR gate when applied to the eigenstates of the \textit{same} basis. Whilst using the state copy rule (\ref{state-copy}), we can show that a spider with one input and two outputs behaves like a classical COPY gate when applied to the eigenstates of the \textit{opposite} basis.

\begin{figure}[H]
    \centering
    \includegraphics[width=0.7\textwidth]{chapter-2/xor_copy}
    \caption{XOR gate (left) and COPY gate (right) with respect to the $Z$ eigenstates.}
    \label{xor}
    \label{copy}
\end{figure}

The natural commutation relation of the classical XOR and COPY gates, implies the \textit{bialgebra rule} (\textit{ba}). By successively applying the two-input/two-output case, we can generalise to any number of inputs and outputs.

\includezxdiagram{chapter-2/bialgebra}{1}

\begin{figure}[H]
    \centering
    \includezxdiagram{chapter-2/bialgebra2}{0.5}
    \caption{The bialgebra rule.}
    \label{bialgebra}
\end{figure}


%%%

\subsection{Hopf Rule}

Like with the bialgebra rule, our motivation for this rule stems from the behaviour of the classical XOR and COPY gates. Since copying two bits then taking their XOR invariably yields 0, we define the \textit{Hopf rule} (\textit{hpf}).

\begin{figure}[H]
    \centering
    \includezxdiagram{chapter-2/hopf}{1}
    \caption{The Hopf rule.}
    \label{hopf}
\end{figure}

Recall that the CNOT gate is both unitary and Hermitian, and therefore, self-inverse. The Hopf rule allows us to prove this diagrammatically as follows.

\includezxdiagram{chapter-2/cnot_hopf}{0.85}
