\section{Clifford Conjugation}%
\label{clifford-conjugation}

The ZX calculus is a diagrammatic language that makes use of the complementary observables in the $Z$ and $X$ bases. Since both of these bases are Hermitian, we can arbitrarily deform their wires as we see fit. For instance, it is easy to show that the Pauli $Z$ gate ($Z$ spider with phase $\pi$) is Hermitian, $Z = Z^\dagger$, by finding its transpose (converting its inputs into outputs and \textit{vice verse}), then taking its complex conjugate (negating its phase $\pi \rightarrow -\pi = \pi$).

\includezxdiagram{chapter-2/hermitian}{0.55}

We now introduce the single-qubit Clifford gates, $R_Z \brac{\pi}{2}$, $R_Z \brac{3\pi}{2}$, $R_X \brac{\pi}{2}$ and $R_X \brac{\pi}{2}$, as defined in Grier \textit{et al} \cite{Grier2016}, in the ZX calculus.

\begin{figure}[H]
    \centering
    \includezxdiagram{chapter-2/cliffords}{1}
    \caption{Representation of single-qubit Clifford gates in the ZX calculus.}
    \label{clifford-definitions}
\end{figure}


Since the $Y$ basis, unlike the $Z$ and $X$ bases, is \textit{not} Hermitian \cite{Yeung2020}, swapping the inputs and outputs of a $Y$ \textit{Spider} does \textit{not} yield its transpose. Instead, we define rotations in the $Y$ basis by conjugating with the Clifford gates described above.

\begin{figure}[H]
    \centering
    \includezxdiagram{chapter-2/Y_rotation}{0.7}
    \caption{Conjugation of generators in the $Z$ and $X$ bases into the $Y$ basis.}
    \label{pauli-Y}
\end{figure}

Using the fact that the Clifford gates are unitary $C^{-1} = C^\dagger$, we expand our definition of the $Y$ rotation to obtain the following commutation relations \cite{Yeung2020}.


\begin{figure}[H]
    \centering
    \includezxdiagram{chapter-2/Y_rotation2}{0.95}
    \caption{Single-qubit Clifford commutation relations.}
    \label{clifford-commutation}
\end{figure}

