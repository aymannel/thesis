\chapter{ZX Calculus}%
\label{zx-calculus}

In this chapter, we introduce the ZX calculus, a diagrammatic language for reasoning about quantum processes first introduced by Coecke \textit{et al} \cite{Coecke2011}. The ZX calculus is described as \textit{universal} in that it can represent any linear map (any complex matrix of dimension $2^n \times 2^m$). Therefore, any equation involving linear maps that is derivable in multilinear algebra can also be derived in the ZX calculus through its rewrite rules \cite{Poor2023}. It is also considered \textit{sound}, meaning that the diagrammatic rewrite rules preserve the underlying semantics of the linear map represented by a given ZX diagram \cite{Wetering2020}.

In this thesis, we use the scalar-free ZX calculus, meaning that the rewrite rules presented in this chapter are correct up to some global non-zero scalar factor. All equal signs should, therefore, be interpreted as `equal up to a global phase'. This is done for convenience, in a similar way to how we sometimes work with unnormalised wavefunctions. Recalling that the matrix representing our quantum circuit $M$ is proportional to some unitary, $M^{-1} = M^\dagger$, we can later efficiently compute the scalar factor \cite{Wetering2020}.

\hangindent=10pt 
\textbf{Remark} -- \textit{All of the definitions in this chapter also hold for their colour-swapped counterparts, which we have chosen to omit for brevity.}

