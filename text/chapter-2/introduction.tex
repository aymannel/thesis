\chapter{ZX Calculus}%
\label{zx-calculus}

In this chapter, we will introduce the ZX calculus, a diagrammatic language for reasoning about quantum processes first introduced by Coecke \textit{et al} \cite{Coecke2011}. We will introduce the basic generators of the ZX calculus, as well as the rewrite rules that make it a language rather than just notation. 

\subsection{Notation}

The flow of time should be interpreted from left to right. Hence, given a particular ZX diagram, we should interpret the leftmost wires as inputs, and the rightmost wires, as outputs.

\includezxdiagram{chapter-2/time}{0.55}

This thesis uses the scalar-free ZX calculus. That is, whilst we obtain the correct operators and states, we omit their associated scalar factor. All equal signs should, therefore, be interpreted as `equal up to a global phase'.

Note that all of the definitions in this chapter also hold for their colour-swapped counterparts, which we have chosen to omit for brevity.
