\chapter{Conclusion}%
\label{conclusion}

\section{Summary}

% The research presented in this thesis focuses on the diagrammatic representation of fermionic excitation operators using the ZX calculus. Working within the DISCO-VQE framework, our initial goal was to identify a generalised structure for these excitation operators, anticipating that this might enable us to gain deeper insights into the nature of the correlations present in molecular systems. We also aimed to rationalise the expectation energy-equivalent ansätze produced by the DISCO-VQE algorithm, potentially optimising the algorithm by minimising the redundancies in its outputs. Furthermore, we sought to optimise the implementation of the UPS ansatz itself by reducing circuit depth.

% Through our exploration, we demonstrated the expressive power of the ZX calculus in representing and manipulating fermionic excitation operators. By using the diagrammatic rewrite rules afforded by the ZX calculus in Chapter \ref{zx-calculus} and the commutation relations developed in Section \ref{pauli-gadgets}, we were able to identify previously unknown non-trivially commuting excitation operators used in the DISCO-VQE algorithm. We used these commutation relations to replicate the work done by Yordanov \textit{et al} in the ZX calculus framework. In this way, we were able to identify a generalised representation for fermionic excitation operators as controlled rotations. In Chapter \ref{controlled-rotations}, we developed a representation for higher order controlled rotations used in the ZX calculus.

% Our findings showed that by identifying commuting pairs of fermionic excitation operators, we can effectively reduce the number of unique ansätze required to explore the Hilbert space. This reduction not only optimises the algorithm but also enhances our understanding of the underlying quantum states and their correlations. Additionally, the insights gained from the ZX calculus allowed us to propose modifications to the DISCO-VQE algorithm, further improving its efficiency and effectiveness.

% The application of the ZX calculus has proven to be a powerful tool in this endeavour, offering both theoretical insights and practical improvements. Future research can build on these findings to develop more advanced algorithms and techniques, further pushing the boundaries of what is possible in quantum simulation of molecular systems.

%%%

The primary objectives of this thesis were to gain deeper insights into the correlation present in molecular systems, rationalise the redundancies among expectation energy-equivalent ansätze, and optimise the UPS ansatz by reducing circuit depth.

The first four chapters of this thesis lay the groundwork for the findings presented in Chapter \ref{excitation-operators}. We discussed DISCO-VQE as the specific algorithm that forms the framework for the research conducted in this thesis and introduced the UPS ansatz, explaining how it can be used to parametrically explore the Hilbert space of possible states to find approximate solutions for the ground state energy. In Chapter \ref{zx-calculus}, we introduced the ZX calculus framework. Then, in Chapter \ref{pauli-gadgets}, we introduced Pauli gadgets and their representation in the ZX calculus, explaining that Pauli gadgets can be used as the building blocks for the fermionic excitation operators used to simulate fermionic systems.

We outlined the set of rules describing how Pauli gadgets interact with each other in Section \ref{commutation-relations}, and in Section \ref{clifford-commutation-relations}, we developed a diagrammatic method to reproducibly derive the interactions of Pauli gadgets with any Pauli or Clifford gate. In Chapter \ref{controlled-rotations}, we extended the work of Yeung \cite{Yeung2020} on singly-controlled rotations to develop a representation of higher-order controlled rotations in the ZX calculus. We later used this result in our attempt to identify a generalised structure for fermionic excitation operators in the ZX calculus.

In Chapter \ref{excitation-operators}, we address the research objectives outlined above. First, we discuss the representation of fermionic excitation operators in terms of Pauli gadgets in Section \ref{implementing-excitation-operators}, explaining how they can be reliably implemented as quantum circuits with optimal circuit depth. We demonstrate the advantages of this architecture-independent representation and show how the gadget commutation rules derived in Section \ref{commutation-relations} provide insights into the structure of the UPS ansatz.

We discuss the expectation energy-equivalent UPS ansätze generated by the DISCO-VQE algorithm in Section \ref{operator-commutations} and propose a rationalisation for the redundancy in their representation. We identify several previously unknown non-trivially commuting excitation operators in the operator pool and propose a modification to the DISCO-VQE algorithm to account for these redundancies. We demonstrate the generality of this result, explaining how these commutation relations hold for any molecular system with a given number of spin orbitals.

In Section \ref{operator-controlled-rotations}, we use the commutation relations developed in Section \ref{clifford-commutation-relations} to replicate the results by Yordanov \textit{et al} and Kornell \textit{et al} using the ZX calculus. By conjugating the excitation operators with a Clifford subcircuit, we obtain a controlled rotation, whose representation in the ZX calculus we derived in Chapter \ref{controlled-rotations}. These derivations validate the expressive power of studying excitation operators diagrammatically.

In Chapter \ref{zxfermion}, we introduce the software package ZxFermion that we built to explore research ideas related to circuits of Pauli gadgets. By encoding the commutation relations developed previous chapters, we were able to straightforwardly replicate the results of Yordanov \textit{et al} and Kornell \textit{et al}, demonstrating a noteworthy acceleration in research pace. By providing an accessible tool for studying the interactions of Pauli gadgets, we hope to facilitate the research done by future chemists wanting to get started in quantum computing, hence lowering the barrier to entry.

\section{Future Work}

In Section \ref{operator-commutations}, we proposed that certain sequences of fermionic excitation operators could be identified as equivalent to others. This suggests that we might demonstrate the equivalence of UPS ansätze generated by the DISCO-VQE algorithm by replacing specific sequences of operators. Given the universality and completeness of the ZX calculus \cite{Coecke2011}, we propose that the equivalence between two operator sequences could be demonstrated using the ZX calculus's diagrammatic rewrite rules.

Building on the generality of the excitation operator commutation relations discussed in Section \ref{operator-commutations}, we suggest that the DISCO-VQE algorithm can be systematically improved for larger systems by identifying non-trivially commuting excitation operators in large spin orbital systems. To this end, we propose implementing the \lstinline{OperatorPool} class in the ZxFermion software package. This class would generate the set of excitation operators used to construct the UPS ansatz based solely on the number of spin orbitals in the system. Using the commutation rules described in Section \ref{commutation-relations}, the \lstinline{OperatorPool} class would systematically identify which excitation operators commute trivially and which commute non-trivially.

As noted by Cowtan \textit{et al} \cite{Cowtan2020}, a set of commuting Pauli gadgets can always be simultaneously diagonalised by a Clifford subcircuit. The resulting phase polynomial can then be optimally resynthesised, similar to how we expressed excitation operators in terms of controlled rotations in Section \ref{operator-controlled-rotations}, allowing for a more efficient implementation. We propose using the excitation operator commutation relations derived in Section \ref{operator-commutations} to identify sequences of commuting excitation operators in a given UPS ansatz. By finding a Clifford subcircuit to simultaneously diagonalise these operators, we can potentially uncover a more generalised structure or optimally resynthesise it using phase polynomial synthesis algorithms \cite{Amy2013}, \cite{Amy2014}, \cite{Nam2018}.
